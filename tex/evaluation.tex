In this section we demonstrate \PNAME in a series of small experiments and
study its performance characteristics.

\vspace{-0.05in}
\section{Working Set Study}

\begin{figure*}[t]
    \vspace{-0.15in}
    \centering
    \begin{subfigure}[t]{0.49\textwidth}
        \includegraphics[width=\columnwidth]{figures/dram_fidelity_speed_slowdown.pdf}
    \end{subfigure}
    \begin{subfigure}[t]{0.49\textwidth}
        \includegraphics[width=\columnwidth]{figures/dram_fidelity_rate_slowdown.pdf}
    \end{subfigure}
    \vspace{-0.20in}
    \caption{Target-execution time of SPEC2017 intspeed and intrate~(4 copies)
    with reference inputs for DRAM models with and without refresh enabled.
    Runtime is normalized to that of a single-cycle memory system. LLCs, if
    present, are \wunits{256}{KiB} and \wunits{1}{MiB} large for intspeed and
    intrate respectively and are 8-way set associative.}
    \label{fig:model-fidelity-sweep}
    \vspace{-0.10in}
\end{figure*}
\begin{figure}[t]
    \centering
    \includegraphics[width=\columnwidth]{figures/cache_size_bar_plot.pdf}
    \vspace{-0.35in}
    \caption{Speedup in SPEC2017 intspeed~(reference inputs) vs LLC model size. All caches are 8-way set associative.}
    \label{fig:llc-speedup}
    \vspace{-0.15in}
\end{figure}

Caches help insulate processors against long and variable latencies to DRAM.
Unfortunately, large LLCs are difficult to implement directly on FPGAs due to
area limitations. \PNAME enables the architect to explore cache sizes that
would otherwise be impossible to host on a single FPGA. We explore this by
sweeping LLC-cache sizes from \wunits{64}{KiB} to \wunits{1}{MiB} in size. In
Figure~\ref{fig:llc-speedup} we show the speed up provided by different cache
sizes and contrast them against a cache-less and single-cycle memory-system.

\section{Effects of DRAM Limitations}

In this experiment, we quantify the extent to which specific timing details
affect target execution time. We start with the validated model, \emph{Full}, and
gradually strip out features: \emph{No Refresh} disables refresh, \emph{No ACT
Limits} sets $t_{FAW}$ and $t_{RRD}$ to zero, and finally, \emph{Ideal} removes
all other timing considerations\footnote{$t_{RTP}$, $t_{WTR}$,
$t_{RTRS}$, $t_{RTP}$=0; $t_{RAS} = t_{RCD} + t_{CS}$ $t_{RC} =
t_{CS} + t_{RCD} + t_{RP}$.}.

The slowdowns of these models relative to a single-cycle memory system are
shown in Figure~\ref{fig:model-fidelity-sweep}. Without an LLC, the largest
source of slowdown is refresh, which contributes a $1.01\times$ and
$1.11\times$ slowdown for intspeed and intrate, respectively.  Eliminating $t_{RRD}$ and $t_{FAW}$ had almost no effect; we
suspect reducing rank count and increasing device density would induce a perceptible slowdown.
Remaining DRAM non-idealities contribute a $1.02\times$ slowdown.  Once
a cache is included, these slowdowns are effectively mitigated: refresh
contributes a only $1.01\times$ slowdown in intrate.

\vspace{-0.05in}
\section{Simulation Performance}

\begin{table*}[t]
\tabcolsep=0.10cm
\centering
    \begin{tabular}{l|rr|rr|rr|rr|rr|rr|rr|rr|rr|rr|}
        Benchmark & \multicolumn{2}{c}{perlbench} & \multicolumn{2}{c}{gcc} & \multicolumn{2}{c}{mcf} & \multicolumn{2}{c}{omnetpp} & \multicolumn{2}{c}{xalancbmk} & \multicolumn{2}{c}{x264} & \multicolumn{2}{c}{deepsjeng} & \multicolumn{2}{c}{leela} & \multicolumn{2}{c}{exchange2} & \multicolumn{2}{c}{xz} \\
Model &      hrs &   MHz &  hrs &   MHz &  hrs &   MHz &    hrs &   MHz &      hrs &   MHz &  hrs &   MHz &      hrs &   MHz &  hrs &   MHz &      hrs &   MHz &  hrs &  MHz \\
\hline
\hline
Single Cycle &      14.4 &   95.8 &  20.4 &   73.3 &  24.7 &   62.1 &    13.8 &   67.4 &      14.0 &   68.6 &  12.9 &  123.0 &      13.4 &   87.5 &   8.6 &  119.0 &       6.9 &  153.3 &  50.5 &  90.5 \\
        FCFS, 256KB LLC     &      14.7 &  100.4 &  20.8 &  91.4 &  25.6 &  102.7 &    14.1 &  86.1 &      14.5 &  88.4 &  13.1 &  125.2 &      13.6 &  91.5 &   8.6 &  121.2 &       6.9 &  153.0 &  51.8 &  105.8 \\
        FCFS, No LLC     &      14.7 &  126.4 &  20.9 &  113.1 &  25.7 &  112.2 &    14.3 &  119.6 &      14.7 &  118.1 &  13.2 &  137.7 &      13.6 &  117.6 &   8.7 &  135.3 &       7.0 &  152.2 &  52.1 & 110.6 \\
\hline
\end{tabular}
    \caption{Simulation times and rates for SPEC2017 intspeed running on a
    single-core Rocket Chip target. The models are sorted by
    target-memory-system latency. }
    \label{tbl:intspeed-simtimes}
\vspace{-0.1in}
\end{table*}
\begin{table*}[t]
\tabcolsep=0.10cm
\centering
\begin{tabular}{l|rr|rr|rr|rr|rr|rr|rr|rr|rr|rr|rr|}
Benchmark & \multicolumn{2}{c}{perlbench} & \multicolumn{2}{c}{gcc} & \multicolumn{2}{c}{mcf} & \multicolumn{2}{c}{omnetpp} & \multicolumn{2}{c}{xalancbmk} & \multicolumn{2}{c}{x264} & \multicolumn{2}{c}{deepsjeng} & \multicolumn{2}{c}{leela} & \multicolumn{2}{c}{exchange2} & \multicolumn{2}{c}{xz} \\
Model &      hrs &   MHz &  hrs &  MHz &  hrs &   MHz &    hrs &  MHz &      hrs &   MHz &  hrs &   MHz &      hrs &   MHz &  hrs &   MHz &      hrs &   MHz &  hrs &   MHz \\
\hline
\hline
Single Cycle &      31.6 &   50.5 &  28.5 &  38.5 &  33.3 &   36.0 &    37.1 &  35.5 &      36.6 &   37.6 &  20.8 &   79.8 &      25.8 &   44.4 &  14.9 &   73.3 &       7.0 &  151.3 &  22.6 &   51.8 \\
FR-FCFS, 1MB LLC  &      31.2 &  54.9 &  24.4 &  57.1 &  23.9 &  72.0 &    32.5 &  49.9 &      31.4 &  54.6 &  20.4 &  84.0 &      24.8 &  48.7 &  14.3 &  78.0 &       7.1 &  150.3 &  20.7 &  62.7 \\
FR-FCFS, No LLC  &      21.6 &  111.7 &  19.6 &  98.8 &  20.2 &  104.8 &    27.3 &  96.8 &      22.8 &  110.6 &  15.9 &  125.6 &      15.4 &  110.5 &  11.4 &  120.6 &       7.0 &  152.3 &  16.4 &  107.3 \\ \hline
\end{tabular}
    \caption{Simulation times and rates for SPEC2017 intrate benchmarks with
    four copies running on a quad-core Rocket Chip target.}
    \label{tbl:intrate-simtimes}
\vspace{-0.1in}
\end{table*}


Table~\ref{tbl:intrate-simtimes} gives the
host-execution times and execution speeds for a handful of SPEC2017 runs. The
targets with DDR3 models consistently run at rates above \wunits{100}{MHz}. Theoretically,
\PNAME instances can operate at the host frequency~(160 MHz) if the functional model can always serve timing model requests in time.
On our host, reads and writes take on average 47
and 37 cycles, respectively, when unloaded.  These latencies have
a significant effect on simulator performance when modeling a single-cycle memory
system as that host latency is fully exposed to the simulator, but
they have relatively little impact
when modeling a realistic DRAM system
whose latency~(cycles) is similar to that of the host.
Instances with LLCs fall between these extremes: LLC hits are fast in
target time and thus expose the host-DRAM latency to simulator, while
misses present sufficient target-latency to hide the host-DRAM access.
For our DDR3-2133 target memory system, reads take 20, 34, and 48 cycles for
row hits, closed row, and row misses, respectively. These latencies are large enough that many accesses can be
served by the functional model before the timing model requires them, eliding simulator stalls.
Ironically, modeling a slower DDR-speed-grade slows down the simulator: since $t_{CS}$ is smaller, reads complete in fewer target
\emph{cycles} despite taking more target \emph{time}.

\vspace{-0.05in}
\section{Instrumentability}
\PNAME allows the user to instrument timing models without changing target
behavior. Coupled with fast simulation speeds, this allows a \PNAME instance to
provide insight into system-wide behavior that would be difficult to collect
otherwise. We demonstrate this in Figure~\ref{fig:leela-time-series}, where we
see how row-buffer and LLC hit rates are inversely correlated through each of
641.leela's games of Go.

\begin{figure*}
    \vspace{-0.15in}
    \centering
    \includegraphics[width=\textwidth]{figures/leela-time-series.pdf}
    \vspace{-0.30in}
    \caption{CPI, D\$ MPKI, and row buffer and LLC hit rates running
    641.leela\_s with on Rocket with \wunits{256}{KiB} of LLC and a FCFS MAS
    model. These plots use a rolling average of 10 samples spaced a billion
    cycles apart.}.
    \label{fig:leela-time-series}
\vspace{-0.25in}
\end{figure*}

