While specialization appears to be only path towards higher performance and more
energy efficient computer hardware, the enormous non-recurring engineering (NRE) cost
of designing modern systems-on-a-chip~(SoCs) is a major barrier to the wider
adoption of custom silicon. As part of a larger effort exploring more
cost-effective agile methodologies for designing custom silicon, this work
contributes to a novel \emph{single-FPGA} hardware emulation framework, FireSim, that aims to
radically reduce the cost of doing fast and accurate full-system simulation.

In this dissertation, we start by introducing a compiler infrastructure, called
\texttt{Golden Gate}, capable of performing general multi-cycle resource
optimizations in order to fit larger SoCs on a single FPGA. The
nature of these optimizations is described at length in A. Magyar's
dissertation~\cite{MagyarDissertation}. Using this compiler infrastructure, we
study optimization-compatible schemes for simulating SoCs with more realistic
clocking organizations.  We first present a simple approach for simulating
systems with multiple fixed-frequency clocks.  We then extend this to support a
more general class of clock switching and generation behavior, notably to
enable timing-exact simulation dynamic frequency scaling. Our approach is based
on prior work in software-based, conservative parallel-discrete event
simulation wherein we replace clock switching and generation structures, like
clock multiplexors, with decoupled models that act on timestamped message
streams.  Unlike other academic FPGA-based systems, which tend to be FPGA
prototypes that rely on direct instantiation of FPGA clocking primitives, here
we strictly use clock gating to derive simulated clocks, making our approach
far easier to use and FPGA portable.
