To avoid confusion when speaking of computers simulating computers, the
literature commonly makes a distinction between the \emph{target}, the system
being simulated, and the \emph{host} or \emph{host-platform}, the system
executing the simulation. The host-platform is often not a single machine but
a collection of interconnected machines, which may include CPUs,
GPUs, and FPGAs.

\section{A Tour of Full-System Simulation}

Simulation is central in performing three fundamental tasks of SoC design.

\begin{enumerate}

    \item \textbf{Prototyping:} ``What thing should we
        build?" Prototyping serves as a means to rapidly evaluate different
        design points with an imperfect model of a proposed design.

    \item \textbf{Verification:} ``Did we build the thing right?" Verification
        serves to check, or prove, that a particular implementation
        correctly executes.

    \item \textbf{Validation:} ``Did we build the right thing?" Validation
        serves to show that the implementation fulfills the objectives set out
        for the system.

\end{enumerate}

Both prototyping and verification can be applied at all levels of the design
hierarchy.  For example, given a specification of the system into which an
accelerator is integrated, one could prototype different design points and
verify an implementation of that accelerator. Validation, however, seeks to
answer a system-level question that spans the entire computing stack.  The
surest way to validate a system is not in simulation, but at-speed with a
physical prototype or the final product itself. However, waiting for a silicon
prototype pushes validation late into the design cycle making it challenging or
impossible to pivot the design of the system based on validation results. To
perform \emph{pre-silicon} validation, a fast and accurate full-system
simulator is required.

Here, SoC designers are confronted with a fidelity-performance-cost trade-off,
and are forced use multiple different simulation technologies at different
points in this space. A summary of these technologies can be found in
Table~\ref{tbl:full-system-simulation-tech}.

\subsection{CPU-Hosted Simulation for Prototyping}

Architecture-level simulators such as QEMU\cite{qemu}, which model the system at the
instruction-set-architecture level and include as limited set of standard
device models for I/O, are fast and inexpensive as they run on conventional CPUs.
When augmented with simple timing models, they are ideal for doing initial system prototyping, as
these models can be quickly modified and recompiled.
However, as these timing models become more complex, they become more
challenging to validate, and cruically, the throughput of these simulators rapidly declines.

Continuing in the direction of increasing fidelity, microarchitecture-level
simulators such as Gem5~\cite{gem5}, and MARSSx86~\cite{marssx86}. are CPU-hosted 
simulators that provide configurable, "cycle-level" timing models of a complete systems, including CPU pipelines,
caches, and off-chip memory systems.  These simulators can run target workloads at hundreds of KIPS, but are
often much slower in practice when employing more detailed or custom models. This
makes it practically impossible to run complete workloads, such as
multi-threaded Java applications or SPECint2006~\cite{spec} with its reference
inputs. Here, a common remedy is to employ statistical sampling
techniques~\cite{smarts} to fast-forward to the region of interest on an architecture-level simulator, before
executing O(100M) instructions at the desired fidelity.

While this approach has well-acknowledged shortcomings~\cite{gem5error},
judicious use of cycle-level simulators can be an appropriate vehicle for
doing initial system prototyping. For radical proposals that involve
aggressive microarchitectural changes or traverse multiple layers of the
computing stack, this approach is often inadequate (particularly for workloads that
are multithreaded, or are long-running and irregular, for which it is diffcult to collect 
meaningful samples without peturbing the system under evaluation, such as
managed-language workloads~\cite{MicroSimPanel}).


\subsection{CPU-Hosted Simulation for Verfication}

Since the aforementioned simulation techniques used abstract models of the
target system, they are useless for system verifaction and validation once
implementation begins. (In fact, those models will need to be validated against the
implementation as it is completed). Instead, here designers use CPU-hosted
simulators that faithfully represent the implementation at a particular
abstraction level.

For simulating digital components of the SoC, a Register Transfer Level~(RTL) ,
like Synopsys VCS, or Verilator, is the tool of choice. Broadly speaking, RTL
simulators model state elements and the combinational functions that update
them at cycle boundaries. Values on wires transition instantiously, and in
general no delay through launching registers, combinational circuits is
modelled. Supposing the underlying digital abstraction holds, RTL simulation
ideal for doing dynamic verification of a digital circuit. Small blocks compile
in seconds, while complete SoCs can be compiled in 1s to 10s of minutes. RTL
simulators are relatively easy to debug, as they provide complete visibility
over the state of the design over the entire duration of a simulation. The
cost~(\$) of an RTL simulator comes mainly from licensing fees, as simulators
run on standard servers or desktop CPUs, though in many cases open-source RTL
like Verilator can be used instead. Ultimately, the largest challenge in using
CPU-hosted RTL simulator is that they have poor simulation throughput for large
designs. Completely SoCs execute at hundreds to less than 1 Hz -- much too slow
to do verification for all but small inputs (e.g., checking system boot), or
for doing full-system validation.

It's important to note at this point that, higher-fidelity software simulation
of the design is commonly used after the SoC is synthesized and implemented in
a particular process technology.  These simulations include combinational,
wire, and parasitic delays, and can generally include more detailed models of
analog components of the system. The additional fidelity only exercabates the
throughput limitation of CPU-based simulation.

To do effective full-system validation and dynamic verification, considerably
faster simulation throughput is required. While there are many techniques that
can improve throughput on CPUs, such as multithreading, relaxing or restricting
the the timing sematics of the design language, ultimately the abundant
fine-grained, often bit-level, parallelism of RTL simulation cannot exploited
by multiprocessors to overcome the 6+ order of magnitude slow down over a
silicon prototype.

\subsection{Hardware-Accelerated Full-System Simulation -- FPGA Prototyping}

To build simulators that execute a rates closer to a silicon prototype, desingers turned
to fine-grained parallel hardware to simulate SoCs.  The earliest form of
hardware-accelerated full-system simulation emerged in the late 1980s, and used
programmable logic devices, specifically FPGAs, to directly implement the
design. This is known as \textit{FPGA prototyping}.

Modern FPGA prototypes directly implement the SoC on one more
more FPGAs, often with a custom board design that may include peripherals
identical to those that would be deployed in the final system.  FPGA prototypes
are fast: small prototypes that fit in a single FPGA execute at ones to hundreds of
MHz, while larger prototypes, which must be partitioned across multiple FPGAs,
simulate at hundreds of kHz~\cite{nehalemprototype, atomprototype}.
Often, FPGA prototypes are inexpensive enough that they be can readily
duplicated and distributed across hardware and software engineering teams.

Relative to software simulation, prototypes have greater fixed costs (to buy, design, and/or license the 
prototyping platform), but more critically, are difficult to use:
\begin{enumerate}
    \item Poor design visibility makes it difficult to debug failing systems. Users must instantiate
        FPGA specific debugging hardware which provides only a limited set of
        signals for small window of time, as these tools consume considerable
        FPGA resources.  If the bug is not found on the first iteration, the
        process must be repeated: the design must be resynthesized with a new
        sset of sampled signals.

    \item Long compile times (ones to tens of hours) make it difficult to
        iterate about a design point, and lengthens the aforementioned the debug cycle.

    \item Large designs must be paritioned across multiple FPGAs either
        manually, or with licensed tools. This makes the prototype hardware
        more expensive, and decreases simulation throughput.

    \item Many ASIC structures, such as ASIC memories, and clock generatiors,
        cannot be synthesized to FPGA fabric so must be replaced with an FPGA
        equivalent.

    \item FPGA-specific I/O models are required to build out a complete system. Using hardened
        IP on the FPGA may not be a good model of the target system. Doing software co-simulation of IO
        often reduces simulation throughput.

    \item Prototypes are not natively deterministic making it difficult to
        reproduce certain classes of system failure, especially those that
        involve IO.

    \item Prototypes require complete RTL implementation of the design
\end{enumerate}


One difficulty with using FPGA prototypes as full-system simulators is that they require complete
RTL implementation of the design. While
FPGA prototypes can accelerate software development by months, they become
useful too late in the design cycle to do hardware-software co-design, as the
target hardware cannot easily be modified which limits the scope of potential
changes.

While this approach has well-acknowledged shortcomings~\cite{gem5error},
judicious use of cycle-level simulators can be an appropriate vehicle for
proposing new microarchitectural ideas. But for radical proposals that involve
aggressive microarchitectural changes or traverse multiple layers of the
computing stack, this approach is inadequate (particularly for workloads that
are long-running, irregular and require a large number of cores, such as
managed-language workloads~\cite{MicroSimPanel}). \TODO{Table~\ref{tbl:full-system-simulation-tech} constrasts the speed, fidelity, and cost of the aforementioned simulation technologies.}

\begin{sidewaystable}
\begin{center}
\resizebox{\textwidth}{!}{%
    \begin{tabular}{|p{0.1\textwidth}|p{0.1\textwidth}|p{0.2\textwidth}|p{0.2\textwidth}|p{0.2\textwidth}|p{0.2\textwidth}|}
    \hline
        \textbf{Technology} & \textbf{Examples} & \textbf{Speed}\newline(Relative to Silicon) &
        \textbf{Fixed Cost} \newline(\$ per simulator) &
        \textbf{Variable Cost}\newline(\$ per target-second) & \textbf{``Compile" Time}\newline(Hours)  \\
    \hline
    \hline
        Arch SW Simulator & QEMU \newline Spike & $10^{-1}$ &
        Free + $10^3$ & \TODO{} & 0 - 0.1 \\
    \hline
        $\mu$Arch SW Simulator & Gem5 \newline MARSSx86 & $10^{-6} - 10^{-4}$ &
        Free + $10^{3}$ & \TODO{} & 0 - 0.1 \\
    \hline
        RTL Simulation & VCS \newline Verilator  & $10^{-7} - 10^{-4}$ &
        $10^{3} /seat/yr + 10^{3}$ \newline $10^{3}$ & \TODO{} & $10^{-2} - 10^{-1}$ \\
    \hline
        Single-FPGA prototype & FPGA-zynq & $10^{-2} - 10^{-1}$ & 2495~(ZC706) \newline 495~(Zedboard) &%
        \TODO{10W (ZC706)} & 0.5 - 1 (FPGA-zynq)\newline $10^{0} - 10^{1}$ \\
    \hline
        Multi-FPGA prototype & \cite{nehalemprototype}, \cite{atomprototype} \newline Protium S1  &
        $10^{-4} - 10^{-3}$ & \TODO{} & \TODO{} & \TODO{} \\
    \hline
        Hardware Emulation & Palladium Z1 & $10^{-3} - 10^{-2}$ &
        $10^{6}$ & \TODO{$10^{-2}$} & 140 MG/hr \\
    \hline
        Silicon Test-chip & \TODO{} & 1 & $10^{5} - 10^{7}$ & \TODO{$10^{-5} - 10^{-4}$} & $10^3 - 10^4$ \\
    \hline
\end{tabular}}
\end{center}
    \caption{Constrasting different technologies for building full-system
    simulators; ordered approximately from top-to-bottom in descending
    fidelity. We define ``compile time" to be the time it takes to make one
    design iteration less the time spent in simulation and implementing a design
    change; the time to generate a simulator from a specification of the
    target.}
\label{tbl:full-system-simulation-tech}
\end{sidewaystable}%
The ideal full-system simulator -- sufficient for both academic and industrial
uses -- would be inexpensive, fast and as accurate as desired throughout the
whole design process; it would always possible to run the software stack on a
model of the target hardware at speeds fast enough for software development.
Initially, this simulator could be used for system-level prototyping and design
space exploration, but as desired, more detailed models or RTL implementations
of target components would be integrated. Ultimately, once all of the target
RTL has been integrated, the simulator subsume the role of an FPGA prototype.
Academics need not completely implement an SoC: they may simply stop adding
fidelity to the simulation once they are content with the quality of their
results.

Presently, FPGAs are the only commercial off-the-shelf (COTS) technology
capable of supporting fast, scalable, cycle-accurate simulation. Thus, we
believe any attempt to build this ideal simulator must necessarily use FPGAs.
However, this requires a more flexible perspective of how FPGAs can be deployed as
simulation \emph{accelerators} and not merely RTL emulation devices as they are
used in FPGA prototypes.

%\section{Defining FPGA-Accelerated Simulation}
%
%The distinction between FPGA-accelerated simulation and FPGA prototyping can be
%nebulous: we argue that FPGA prototyping represents a narrow subset of the larger
%space of FPGA-accelerated simulation. Key to understanding the distinction is
%first, think of simulation as any other application executing on a host, and
%second, forget that a host may be or include one or more FPGAs.
%
%A simulation is an application that takes an input and produces an output.  The
%output may be the console or file I/O of the target.  Alternatively, it may be
%a dump of the microarchitectural state of the target as it changes over the
%lifetime of the simulation.  As far as the user is concerned, the simulation
%may be optimized in any way whatsoever so long as this output is the same.
%
%Like any other application, simulations have hot spots that account for the bulk
%of their runtime. To improve runtime, the simulation may be parallelized over
%the host, either over multiple homogeneous resources, or by offloading specific
%kernels to accelerators, which may execute concurrently with the rest
%of the application.
%
%%In simulations that account for time at the cycle-level, much of this runtime
%%is dedicated to modeling the cycle-by-cycle interactions of parts of the
%%system. These models may be written in C++ or SystemC, or implemented in an HDL
%%like verilog or VHDL. Parts of the simulation may be divided into functional
%%models, like an architecture simulator, and a timing-model that does some
%%accounting of time based on a model of the microarchitecture. What's important
%%to note, is that any cycle-level model of hardware attempts to capture the
%%behavior of a fine-grained highly concurrent digital-circuit. Taken to the
%%limit, a cycle-level model is an RTL model.
%
%In FPGA-accelerated simulation, in general, we attempt to offload parts of the
%simulation that do cycle-level or cycle-accurate modeling of some part of the
%target. One way to achieve this is to dissolve the simulation spatially
%(perhaps along module boundaries): parts of the target, like an NoC or
%processor pipeline, could be offloaded to an FPGA, while models for I/O may be
%hosted on a CPU.  Alternatively, one could host a functional model of a module
%on the CPU and accelerate a detailed cycle-accurate timing-model on the FPGA
%(or vice versa).  Again, the only constraint on the implementation of the
%simulation, and thus, the implementation of FPGA-accelerated components of the
%simulation, is that its output remains the same. All other things equal, a
%faster implementation is always better.
%
%\section{Compelling Ideas in FPGA-accelerated Simulation}
%
%Researchers have devised many clever techniques for using FPGAs as simulation
%accelerators. To highlight some of these techniques, we summarize the FAME
%taxonomy~\cite{fame}, which outlines three dimensions along which
%FPGA-accelerated simulators may be categorized.
%
%\subsection{FAME-1 (XX1): Host-Target Decoupling}\label{sec:fame1}
%
%In host-decoupled fpga-accelerated simulators, a target-cycle of simulation
%executes over a variable number of FPGA-host cycles. In contrast, an
%FPGA protoype executes a single target-cycle on every FPGA-host cycle. With
%this technique, ASIC structures that map inefficently to FPGA fabric may be replaced
%with optimized-for-FPGA structures that take more host cycles to execute, but save
%FPGA resources and improve host-cycle time.  One classic optimization replaces
%multi-ported register files and CAMs with a dual-ported BRAMs pumped over
%multiple cycles.  Additionally, host-decoupling permits the simulator to
%tolerate variable latencies in the host-platform without sacrificing simulator
%performance or changing the target-time behavior of the simulation. Nearly all
%academic FPGA-accelerated simulators are FAME-xx1 simulators, though
%ProtoFlex~\cite{protoflex} is an early example.
%
%\subsection{FAME-2 (X1X): Abstract RTL}
%
%In an abstract-RTL FPGA-accelerated simulation, components of the simulator do
%not model the implementation RTL exactly. Abstraction permits simplifying
%components of the simulator, trading simulation fidelity for FPGA-resource
%savings. Additionally, abstract models can be made reconfigurable in ways the
%implementation RTL cannot. Most academic FPGA-simulators are to some degree
%abstract-RTL simulators.  For example, Strober~\cite{strober} used a
%fixed-latency pipe to model the target's DRAM subsystem~(a FAME-011 model),
%but used FAME-001 models derived directly from source RTL everywhere else.
%Conversely, RAMPGold~\cite{rampgold} was entirely abstract.
%
%\subsection{FAME-4 (1XX): Multithreading}
%
%In a multithreaded FPGA-accelerated simulation, multiple virtual instances of a
%block or module within the target are simulated using a single physical
%datapath on the FPGA. The target state is duplicated according to the number of
%virtual instances, and a scheduler selects which virtual instance should be
%simulated in a given host-cycle. ASIC logic tends to be expensive when mapped
%to FPGA fabric; in FPGA prototypes, designs tend to be logic~(LUT) constrained,
%which leaves much of the FPGA's embedded BRAM left unused.  Multithreading
%improves the mapping efficiency of the target, by reusing the expensive logic
%over multiple copies of target state which can be mapped into abundant FPGA BRAMs and registers.
%HASim~\cite{hasim} and RAMPGold~\cite{rampgold} are examples of simulators that
%employ FAME-x1x multithreading.
%
%
%\subsection{Split Timing-Functional Models}
%
%Splitting timing and functional models is ubiquitous both in software and
%FPGA-accelerated simulators, as it permits amortizing the design effort of the
%functional model over multiple timing models. This is especially crucial in
%FPGA-based simulation, where the functional model can account for the bulk of
%the complexity (e.g., 35K vs. 1K SystemVerilog LoC in RAMPGold~\cite{rampgold}. With
%host-decoupling, it is possible to host the two models on different parts of
%the host-platform: FAST~\cite{fast} hosted its functional model in software and its
%timing model in fabric (ProtoFlex did the opposite~\cite{protoflex}).
%
%For a comphrehensive survey of FPGA-accelerated simulation work we direct
%the reader to~\cite{fpgasimbook}.
%
%\section{Adoption Challenges}
%
%Despite their promise, FPGA-accelerated simulators have only been employed by
%the researchers that developed them. The failure to adopt FPGA-accelerated
%simulation methodologies more widely comes as a result of several key factors:
%
%\begin{enumerate}
%
%    \item \textbf{Availability.} Much of the early FPGA-accelerated simulation research
%        relied on boutique FPGA-host platforms like the BEE~\cite{bee2}, or
%        used custom board designs. The cost of these platforms disincentivizes
%        their adoption by researchers who already have the means to run
%        software simulations at low cost.
%
%    \item \textbf{FPGA Capacity.} Common ASIC structures, such as CAMs,
%        multi-ported RAMs, and wide multiplexors are known to map poorly to
%        FPGA fabrics~\cite{fpgagap, fpgagap2}, making it difficult to host
%        large target designs on an FPGA.
%
%    \item \textbf{Configurability \& Extensibility.} Extending FPGA-accelerated
%        simulations requires writing RTL. RTL models are less configurable and
%        harder to extend than software models. Finally, RTL models still need
%        to be validated, further exacerbating the challenge of building them.
%
%    \item \textbf{FPGA compile time.} Compiling an FPGA simulator considerably
%        longer than compiling a software simulator (ones of hours).
%        %To some extent this is inescapable. However, where abstract models are
%        %employed, they can be made run-time configurable, with programmable
%        %registers sitting on a simulation memory map. Where models are
%        %generated from RTL, they can can be incrementally recompiled, or perhaps
%        %partially reconfigured.
%
%    \item \textbf{Debuggability.} Debugging a broken FPGA-accelerated
%        simulation is difficult due to the limited visibility the designer has
%        over the state of the simulation. This is often more challenging than
%        debugging an FPGA prototype of the target, as FPGA-specific
%        optimizations make it more difficult to reason about the state of the
%        target.
%
%\end{enumerate}
%
%\section{Why Continue With FPGA-Accelerated Simulation?}
%
%Even as Moore's law wanes, FPGA capacity continues to scale. The largest FPGAs
%have over 50 MB of BRAM and millions of logic cells\footnote{Comically, scaling
%RAMPGold~\cite{rampgold}, to use the largest Xilinx UltraScale
%FPGA~\cite{ultrascale} by BRAM capacity would permit modeling in excess of 5000
%cores.}. As they have scaled, FPGAs have continued to become more
%heterogeneous, adding features that make them more amenable to hosting
%full-system simulaions.  Both Intel and Xilinx now sell FPGAs with embedded ARM
%cores, making it easier to co-simulate tightly coupled hardware and software
%models of a system. Modern FPGAs include hardened DRAM controllers that are
%comparable to those of ASICs. This trend towards greater integration looks to
%continue. For example, upcoming Intel Stratix 10 MX parts include in-package DRAM (HBM2)
%that can support up to 1 TBps of aggregate memory bandwidth~\cite{stratix10mx}.
%Both DRAM capacity and bandwidth are crucial for simulating components
%of the target that may not fit in BRAM.
%
%Lower cost and increased on-chip integration have also made FPGAs more
%accessible. Not only are COTS development boards cheaper and more featureful,
%FPGAs are now available in academic clusters, like TACC's Microsoft
%Catapult~\cite{catapultannounce} deployment, and in datacenters, as a service
%like Amazon Web Services' EC2 F1 instances~\cite{amazonf1}. Where in the past
%academics would have to purchase their own FPGAs -- even to reproduce published
%experiments -- it may soon be possible for them to instead spin up an identical
%simulation on a shared computing resource. Companies may no longer have to
%maintain their own FPGA prototyping clusters; they could instead batch out
%simulations to the cloud.
%
%\section{Improving Usability Through Automation}
%
%While the trends described in the previous section solve the
%\emph{availability} and \emph{FPGA-capacity} challenges, the remaining three
%usability challenges persist. Previous work~\cite{fabscalarfpga, strober} has
%shown that much of an FPGA-accelerated simulator can be automatically generated
%from source RTL. This RTL can be written in an HDL like Verilog or emitted by
%high-level synthesis tools or generators written in languages like
%Chisel~\cite{chisel}. While this still requires an RTL implementation, the same
%RTL can be passed to an EDA flow to measure physical design metrics, and,
%best of all, no validation of the generated model is required.
%
%This is, unfortunately, not a panacea: perhaps the RTL is not yet
%available, or a more abstract, reconfigurable model is desired. In this report
%we consider off-chip DRAM memory systems as a motivating example: they have too
%much state to be hosted in-fabric and yet they must must be tightly coupled to
%the processor model. This makes it difficult to co-simulate DRAM in
%software.\footnote{High-latency peripherals, like disks, can often be modeled
%in software without any performance cost~\cite{disksim}.} These components
%typically require an abstract model that virtualizes the target-memory system
%over FPGA-host-memory system.
%
%This reintroduces the aforementioned problem that anything but a simplistic RTL
%model is difficult to design, modify and reuse. We propose to address this
%through \emph{generators} that synthesize abstract memory system models that
%can be easily modified and used across a wide range of targets.  This generator
%must provide a variety of timing models so as to enable the designer to trade
%fidelity for FPGA-resource savings when needed. Generated models must be reconfigurable, to
%permit sweeping memory system parameters without needing to recompile the FPGA
%bitstream. These models must provide useful instrumentation to both aid in
%debugging and to provide insight about memory system behavior without
%perturbing execution. Finally, when the generator does not provide an
%appropriate timing model, the generator should be easy to extend.
