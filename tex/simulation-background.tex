Full-system simulation is used ubiquitously in SoC design and verification. In
the first third of this chapter, we explore the roles full-system simulation
serves in the SoC design process, and discuss commonly used full-system
simulation technologies with emphasis on hardware-accelerated forms like FPGA
prototyping and hardware emulation.
In the second third, we study how computer architecture researchers employ full-system simulation
conduct higher-level architectural and microarchitectural studies, and how academics have
sought to use FPGAs to accelerate these simulations.
We conclude by reviewing work done by the Berkeley Architecture Research~(BAR)
group over the past decade that attempts to build more cost-effective hardware
emulators by applying techniques from these FPGA-accelerated microarchitecture
simulators as automatic transformations to SoC RTL.

To avoid confusion when speaking of computers simulating computers, the
literature commonly makes a distinction between the \emph{target}, the system
being simulated, and the \emph{host}, the system
executing the simulation. The host is often not a single machine but
a collection of interconnected machines, which may include CPUs,
GPUs, and FPGAs.

\section{A Tour of Full-System Simulation for SoC Design}

Simulation is central to performing three fundamental tasks of SoC design.

\begin{enumerate}

    \item \textbf{Prototyping:} ``What thing should we
        build?" Prototyping serves as a means to rapidly evaluate different
        design points with an imperfect model of a proposed design.

    \item \textbf{Verification:} ``Did we build the thing right?" Verification
        serves to check, or prove, that a particular implementation
        correctly executes.

    \item \textbf{Validation:} ``Did we build the right thing?" Validation
        serves to show that the implementation fulfills the objectives set out
        for the system.

\end{enumerate}

Both prototyping and verification can be applied at all levels of the design
hierarchy.  For example, given a specification of the system into which an
accelerator is integrated, one could prototype different design points and
verify an implementation of that accelerator. Validation, however, seeks to
answer a system-level question that spans the entire computing stack.  The
surest way to validate a system is not in simulation, but at-speed with a
physical prototype or the final product itself. However, waiting for a silicon
prototype pushes validation late into the design cycle making it challenging or
impossible to pivot the design of the system based on validation results. To
perform \emph{pre-silicon} validation, a fast and accurate full-system
simulator is required.

Here, SoC designers are confronted with a fidelity-performance-cost trade-off,
and are forced use multiple different simulation technologies at different
points in this space. A summary of these technologies can be found in
Table~\ref{tbl:full-system-simulation-tech}.

\subsection{CPU-Hosted Simulation for Prototyping}\label{UArchSWSim}

Architecture-level simulators such as QEMU\cite{QEMU}, which model the system at the
instruction-set-architecture level and include as limited set of standard
device models for I/O, are fast and inexpensive as they run on conventional CPUs.
When augmented with simple timing models, they are ideal for doing initial system prototyping, as
these models can be quickly modified and recompiled.
However, as these timing models become more complex, they become more
challenging to validate, and crucially, the throughput of these simulators rapidly declines.

Continuing in the direction of increasing fidelity, microarchitecture-level
simulators such as Gem5~\cite{Gem5}, and MARSSx86~\cite{MARSSx86}. are CPU-hosted 
simulators that provide configurable, "cycle-level" timing models of a complete systems, including CPU pipelines,
caches, and off-chip memory systems.  These simulators can run target workloads at hundreds of KIPS, but are
often much slower in practice when employing more detailed or custom models. This
makes it practically impossible to run complete workloads, such as
multi-threaded Java applications or SPECint2006~\cite{SPEC2006} with its reference
inputs. Here, a common remedy is to employ statistical sampling
techniques~\cite{smarts} to fast-forward to the region of interest on an architecture-level simulator, before
executing O(100M) instructions at the desired fidelity.

While this approach has well-acknowledged shortcomings~\cite{gem5error},
judicious use of cycle-level simulators can be an appropriate vehicle for
doing initial system prototyping. For radical proposals that involve
aggressive microarchitectural changes or traverse multiple layers of the
computing stack, this approach is often inadequate (particularly for workloads that
are multithreaded, or are long-running and irregular, for which it is difficult to collect 
meaningful samples without perturbing the system under evaluation, such as
managed-language workloads~\cite{MicroSimPanel}).


\subsection{CPU-Hosted Simulation for Verification}

Since the aforementioned simulation techniques use abstract models of the
target system, they are useless for system verification and validation once
implementation begins. (In fact, those models will need to be validated against the
implementation as it is completed). Instead, here designers use CPU-hosted
simulators that faithfully represent the implementation at a particular
abstraction level.

For simulating digital components of the SoC, a Register Transfer Level~(RTL) ,
like Synopsys VCS, or Verilator, is the tool of choice. Broadly speaking, RTL
simulators model state elements and the combinational functions that update
them at cycle boundaries. Values on wires transition instantaneously, and in
general no delay through launching registers, combinational circuits is
modeled. Supposing the underlying digital abstraction holds, RTL simulation
ideal for doing dynamic verification of a digital circuit. Small blocks compile
in seconds, while complete SoCs can be compiled in 1s to 10s of minutes. RTL
simulators are relatively easy to debug, as they provide complete visibility
over the state of the design over the entire duration of a simulation. The
cost~(\$) of an RTL simulator comes mainly from licensing fees, as simulators
run on standard servers or desktop CPUs, though in many cases open-source RTL
like Verilator can be used instead. Ultimately, the largest challenge in using
CPU-hosted RTL simulator is that they have poor simulation throughput for large
designs. Completely SoCs execute at hundreds to less than 1 Hz -- much too slow
to do verification for all but small inputs (e.g., checking system boot), or
for doing full-system validation.

It's important to note at this point that, higher-fidelity software simulation
of the design is commonly used after the SoC is synthesized and implemented in
a particular process technology.  These simulations include combinational,
wire, and parasitic delays, and can generally include more detailed models of
analog components of the system. The additional fidelity only exacerbates the
throughput limitation of CPU-based simulation.

To do effective full-system validation and dynamic verification, considerably
faster simulation throughput is required. While there are many techniques that
can improve throughput on CPUs, such as multithreading, relaxing or restricting
the the timing semantics of the design language, ultimately the abundant
fine-grained, often bit-level, parallelism of RTL simulation cannot exploited
by multiprocessors to overcome the 6+ order of magnitude slow down over a
silicon prototype.

\subsection{FPGA Prototyping}\label{sec:fpga-prototyping}

To build simulators that execute a rates closer to a silicon prototype, designers turned
to fine-grained parallel hardware to simulate SoCs.  The earliest form of
hardware-accelerated full-system simulation emerged in the 1980s, and used
programmable logic devices, specifically FPGAs, to directly implement the
design. This is known as \emph{FPGA prototyping}.

Modern FPGA prototypes directly implement the SoC on one more
more FPGAs, often with a custom board design that may include peripherals
identical to those that would be deployed in the final system.  FPGA prototypes
are fast: small prototypes that fit in a single FPGA execute at ones to hundreds of
MHz, while larger prototypes, which must be partitioned across multiple FPGAs,
simulate at hundreds of kHz~\cite{nehalemprototype, atomprototype}.
Often, FPGA prototypes are inexpensive enough that they be can readily
duplicated and distributed across hardware and software engineering teams.

Relative to software simulation, prototypes have greater fixed costs (to buy, design, and/or license the 
prototyping platform), but more critically, are difficult to use:
\begin{enumerate}
    \item Poor design visibility makes it difficult to debug failing systems. Users must instantiate
        FPGA specific debugging hardware which provides only a limited set of
        signals for small window of time, as these tools consume considerable
        FPGA resources.  If the bug is not found on the first iteration, the
        process must be repeated: the design must be resynthesized with a new
        set of sampled signals.

    \item Long compile times (ones to tens of hours) make it difficult to
        iterate about a design point, and lengthens the aforementioned the debug cycle.

    \item Large designs must be partitioned across multiple FPGAs either
        manually, or with specialized tools. This makes the prototype hardware
        more expensive and decreases simulation throughput.

    \item Many ASIC structures, such as multi-ported RAMs and clock generators,
        cannot be synthesized to FPGA fabric so must be replaced with an FPGA
        equivalent.

    \item FPGA-specific I/O models are required to build out a complete system. Using hardened
        IP on the FPGA may not be a good model of the target system. Doing software co-simulation of IO
        often reduces simulation throughput.

    \item Prototypes are not natively deterministic making it difficult to
        reproduce certain classes of system failure, especially those that
        involve IO.

    \item Prototypes require complete RTL implementation of the design
\end{enumerate}

These limitations make FPGA prototypes unproductive for doing full-system
verification. However, in most cases FPGA prototypes are the fastest available
full-system RTL model the target and so are used extensively for
pre-silicon software development and for regression testing.

\subsection{Hardware Emulation}

Hardware emulation aims to provide productive full-system verification by
marrying the speed FPGA prototypes with the usability software simulators.
Hardware emulators tend to be expensive to license and run -- millions of
dollars per unit -- making them a precious commodity that must be carefully
shared across a company.

Each of the three major CAD vendors offer hardware emulation solutions.
Mentor Veloce~\cite{Veloce} and Cadence Palladium~\cite{Palladium} are the historical market leaders in
hardware emulation, with the two often swapping positions with the release of updated
emulation platforms, with Synopsys ZeBu~\cite{ZeBu} rounding out the offerings.
Differences in the implementations of these three emulators put them at
different points in the cost-usability-performance space -- we describe these three strategies in the following sections.

\subsubsection{ASICs - Logic Processor Arrays}

Perhaps the earliest form of hardware emulation traces back to IBM research's
Yorktown Simulation Engine~(YUE, \cite{YSEHardware}) and it's industrial-strength successor project
the Engineering Verification Engine~(EVE, \cite{EngineeringVerificationEngine}). These machines consisted of an array
of small processors which would simulates pieces of the DUT. Processors were
specialized for simulating logic (Logic Processors) and memory (Array
Processors), and were interconnected with high-radix crossbars (256 x 256 in
these early machines). A compiler~\cite{YSESoftware} would partition, map, and schedule
the DUT onto this array. Both EVE and YUE could do gate-level simulation in zero-delay
or unit-delay (every gate propagation incurs a constant delay) modes. Cyclist~\cite{Cyclist}
is a recent academic work that further explores this approach; it uses a
homogeneous array of RISC-V cores attached in a 2-D mesh network.

Cadence's Palladium~\cite{Palladium} emulators derive from EVE (which was for a time sold by QuickTurn Design Systems under the name CoBALT~\cite{CoBALT}, before they
were acquired by Cadence. Relative to
competing emulation solutions, Palladiums have lower capacity per unit, but the
fastest compile times. Palladiums also have highest power consumption and must be
water-cooled -- unlike the competing offerings.


\subsubsection{Custom FPGAs}
While a EVE-like processor arrays can provide radically improved compilation speeds since
the compilation problem is fundamentally simpler than running FPGA place and
route, one natural alternative is to modify conventional FPGAs for emulation.
This might involve adding dedicated hardware to improve debuggability (e.g., to make it
possible to capture full-visibility waveforms), adding I/O multiplexing
hardware to ease the FPGA-partitioning problem~(i.e., a hardware implementation of Virtual Wires~\cite{VirtualWires}), and modifying the programmable
fabric to better match the resources and interconnect required ASIC designs.

This is the approach taken by Mentor Graphics's Veloce~\cite{Veloce} emulation platforms.
Veloce emulators appear to have better capacity than Palladium, but slower
compile times. Veloce emulators use less power and are air-cooled.

\subsubsection{Commercial-Off-The-Shelf FPGAs}

Hardware emulation large total-cost of ownership is driven to some degree
because the leading emulators are both power-hungry and use custom silicon. To
skirt some of the costs of those platforms, a third alternative that sees
industrial use is to use large, commercial-off-the-shelf FPGAs but rely on
custom tooling and system packaging to improve usability. Synopsys's
ZeBu~\cite{ZeBu} platform does precisely this by leveraging the largest
available Xilinx FPGAs available at the time of its release (Virtex Ultrascale
VU440s~\cite{ZeBu}). At time of writing, ZeBu provides the highest simulation
throughput and the lowest power consumption but has a reduced debug feature set
relative to other emulators.

In it's use of COTS FPGAs, Synopsys ZeBu is most similar to the work presented herein. 

%\begin{sidewaystable}
%\begin{center}
%\resizebox{\textwidth}{!}{%
%    \begin{tabular}{|p{0.1\textwidth}|p{0.1\textwidth}|p{0.2\textwidth}|p{0.2\textwidth}|p{0.2\textwidth}|p{0.2\textwidth}|}
%    \hline
%        \textbf{Technology} & \textbf{Examples} & \textbf{Speed}\newline(Relative to Silicon) &
%        \textbf{Fixed Cost} \newline(\$ per simulator) &
%        \textbf{Variable Cost}\newline(\$ per target-second) & \textbf{``Compile" Time}\newline(Hours)  \\
%    \hline
%    \hline
%        Arch SW Simulator & QEMU \newline Spike & $10^{-1}$ & Free + $10^3$ & \TODO{} & 0 - 0.1 \\
%    \hline
%        $\mu$Arch SW Simulator & Gem5 \newline MARSSx86 & $10^{-6} - 10^{-4}$ & Free + $10^{3}$ & \TODO{} & 0 - 0.1 \\
%    \hline
%        RTL Simulation & VCS \newline Verilator  & $10^{-7} - 10^{-4}$ & $10^{3} /seat/yr + 10^{3}$ \newline $10^{3}$ & \TODO{} & $10^{-2} - 10^{-1}$ \\
%    \hline
%        Single-FPGA prototype & FPGA-zynq & $10^{-2} - 10^{-1}$ & 2495~(ZC706) \newline 495~(Zedboard) &% \TODO{10W (ZC706)} & 0.5 - 1 (FPGA-zynq)\newline $10^{0} - 10^{1}$ \\
%    \hline
%        Multi-FPGA prototype & \cite{nehalemprototype}, \cite{atomprototype} \newline Protium S1  & $10^{-4} - 10^{-3}$ & \TODO{} & \TODO{} & \TODO{} \\
%    \hline
%        Hardware Emulation & Palladium Z1 & $10^{-3} - 10^{-2}$ & $10^{6}$ & \TODO{$10^{-2}$} & 140 MG/hr \\
%    \hline
%        Silicon Test-chip & \TODO{} & 1 & $10^{5} - 10^{7}$ & \TODO{$10^{-5} - 10^{-4}$} & $10^3 - 10^4$ \\
%    \hline
%\end{tabular}}
%\end{center}
%    \caption{Contrasting different technologies for building full-system
%    simulators; ordered approximately from top-to-bottom in descending
%    fidelity. We define ``compile time" to be the time it takes to make one
%    design iteration less the time spent in simulation and implementing a design
%    change; the time to generate a simulator from a specification of the
%    target.}
%\label{tbl:full-system-simulation-tech}
%\end{sidewaystable}%

\section{Hardware-Accelerated Microarchitecture Simulation}

In order to build more cost-effective hardware emulators, this work draws from
technologies developed in academia to accelerate cycle-level microarchitecture
simulation~(this is described see Section~\ref{UArchSWSim}).

The first, most obvious way to accelerate these is to
parallelize them over multiprocessors or networks of workstations.
One early example of this is the Wisconsin Wind Tunnel~(WWT)\cite{WisconsinWindTunnel}, which relied
on a window-based, parallel discrete-event simulation engine to manage synchronization
across workstations. For CPU-hosted simulators like the WWT, and recent works like Graphite~\cite{Graphite}, to achieve
good performance they must reduce the synchronization overhead between
partitions of the design by either modeling the target more abstractly, or by
introducing extra target-latency between partitions.  While this may be
appropriate for building coarser models, it's ineffective for accelerating RTL
simulations and thus for building hardware emulators, which, at best, require
per-cycle synchronization.

So, in order to build simulators that were both fast and cycle-accurate
many academics turned to FPGAs, which by the 1990s and early 2000s had proven themselves as effective
vehicles for ASIC prototyping and emulation. However, instead of directly implementing an
ASIC design, here FPGAs would be used as a host for microarchitectural models written in RTL.
An early example of this approach is the Rapid Processor Emulator~\cite{RPM, RPMDesign}, however the bulk of the research
in this domain can be attributed to the RAMP project~\cite{RAMP}.

\subsection{Research Accelerator For Multiple Processors (RAMP)}

The RAMP project began in 2005 driven by the realization that advances in computing performance
would require exploiting other forms of parallelism beyond ILP, as the end of
Dennard scaling would necessarily make single-threaded performance improvements
more difficult to attain. The goal was to develop a shared full-system
simulation infrastructure for the computer architecture community which would
be better suited to study future thread-parallel systems of 64 - 1024
processors, than traditional software simulators. Member universities included UT
Austin; CMU; UCB; University of Washington; Stanford; and MIT.
%While
%the RAMP project ultimately failed to produced a unified simulation
%infrastructure, differences in the the simulators produced by the member
%institutions each uniquely advanced the state-of-the-art.

At the onset of the project three initial, monolithic prototypes were built,
each was designed to model a different class of target.
RAMP Red, later known as ATLAS~(Stanford)~\cite{ATLAS} was designed to study transactional-memory-based chip-multiprocessors, and supported up to 8 PowerPC cores.
RAMP Blue~(UCB)\cite{RAMPBlue}) was tailored for large-scale distributed-memory message-passing machines and used Xilinx
Microblaze softcores to prototype the target system. Partitioned over 21
BEE2 boards, RAMPBlue could simulate a system with as many as 1008 cores.
Finally, RAMP White(UT Austin)~\cite{RAMPWhite} modeled cache-coherent
shared-memory processors. It supported both PowerPC (when using a Xilinx
host with a hardened PowerPC 405 core) and 32-bit SPARCV8 (soft core) targets.
Each of these prototypes used the same host platform (the Berkeley Emulation
Engine 2)~\cite{BEE2}, and were initially constructed using shared libraries and a common
specification language called the RAMP design language~\cite{RDL}.

Other, mostly later, projects tied to RAMP abandoned the shared infrastructure
and explored different simulator design styles.
ProtoFlex~(CMU)\cite{ProtoFlex} was an architecture-level simulator that
demonstrated 16-way host-multithreading of a single FPGA-hosted functional
model.  ProtoFlex could switch between FPGA-hosted and CPU-hosted modes via
a process it called transplantation. FAST~(UT Austin)\cite{FAST} was a cycle-accurate
x86 simulator which leveraged a split, CPU-hosted functional model and FPGA-hosted
timing model. RAMPGold~(UCB)\cite{RAMPGold}(UC Berkeley) used FPGA-hosted timing
and functional models with 64-way host-multithreading to realize a larger
target on a single FPGA.  To model a datacenter-scale target,
DIABLO~(UCB)\cite{Diablo} stitched together 24 instances of a modified version RAMPGold to simulate 3072 interconnected
servers.  Finally, HASim~\cite{HASim}(MIT) also used FPGA-hosted timing and
functional models, but provided more detailed pipeline and memory hierarchy
models. Later work studied partitioning HASim over multiple FPGAs.
\cite{LIFPGADesign} and showed that by using two FPGAs HASim could host eight
times as many cores, due to improved resource sharing between virtual
instances.

Around the same period, other groups not associated with the RAMP project explored
using FPGAs for microarchitecture simulation.  One notable example is
DART~\cite{DART}, an FPGA-based Network-on-Chip simulator that used
multithreading, like many RAMP simulators, but leveraged NoC-specific model abstractions
to permit a wide range of runtime-reconfigurable model parameters.

\subsection{FAME Taxonomy}

One output of the RAMP project was the FPGA Architecture Model
Execution~(FAME)\cite{FAME} taxonomization of FPGA-accelerated simulation work,
which distills many of the contributions of the works above into three dimensions: host-decoupling (FAME-1), abstract RTL vs concrete RTL
(FAME-2), and multithreading (FAME-4).

\subsection{FAME-1: Host Decoupling}\label{sec:fame1}

In host-decoupled FPGA simulators, a target cycle of simulation
executes over multiple FPGA-host cycles. In contrast, a
conventional FPGA prototype executes a single target-cycle on every FPGA-host
cycle; multiple clocks can be prototyped using multiple host clocks with the
same relative frequency relationship as exists in the target.
With host-decoupling, ASIC structures that map inefficiently to FPGA fabric may be replaced
with optimized-for-FPGA structures that take more host cycles to execute, but save
FPGA resources and improve host-cycle time.  One classic optimization replaces
multi-ported register files and CAMs with a dual-ported BRAMs pumped over
multiple cycles.  Additionally, host-decoupling permits the simulator to
tolerate variable latencies in the host without sacrificing simulator
performance or changing the target-time behavior of the simulation.
Nearly all academic FPGA-accelerated simulators employ host-decoupling.
Unlike FPGA prototypes, commercial FPGA-based emulators are necessarily host-decoupled to
make them execute deterministically and to support a wealth of additional features that may require
halting parts of the emulation. We expand on mechanisms for implementing host-decoupling in Section~\ref{sim:fame1-abstractions}

\subsection{FAME-2: Abstract RTL}

In an abstract-RTL FPGA-accelerated simulator, components of the simulator do
not model the implementation RTL exactly. Abstraction permits simplifying
components of the target, trading simulation fidelity for FPGA-resource
savings. Additionally, abstract models can be made reconfigurable in ways the
implementation RTL cannot~(i.e., a latency-pipe model can expose it's latency
as a runtime-programmable register).  The FAME-2 dimension of the taxonomy
represents a spectrum in practice.  Most academic FPGA-accelerated simulators
are either partially or completely abstract-RTL simulators.  Even commercial
hardware emulators and FPGA prototypes are to some extent abstract -- as
generally some ASIC features may need to be replaced with an equivalent model
provided by the emulation or prototyping platform.
%Can i cite documentation?

\subsection{FAME-4: Multithreading}

In a multithreaded FPGA-accelerated simulator, like HASim or RAMPGold, multiple virtual instances of a
block or module within the target are simulated using a single physical
combinational datapath on the FPGA. The target state is duplicated according to the number of
virtual instances, and a scheduler selects which virtual instance should be
simulated in a given host-cycle. ASIC logic tends to be expensive when mapped
to FPGA fabric; in FPGA prototypes, designs tend to be logic~(LUT) constrained,
which leaves much of the FPGA's embedded BRAM left unused.  Multithreading
improves the mapping efficiency of the target, by reusing the expensive logic
over multiple copies of target state which can be mapped into abundant FPGA BRAMs and registers.
To the best of our knowledge, commercial hardware
emulators and FPGA prototypes do not using multithreading schemes.

\subsection{RAMP Retrospective}

Ultimately, RAMP-style FPGA-accelerated simulation failed to take off
in the computer architecture community. There are a number of technical explanations for this, though many of these
derive from the aforementioned challenges with using FPGAs to prototype ASICs~(see Section~\ref{sec:fpga-prototyping}).
We expand on three more specific challenges here.

\begin{itemize}
    \item \emph{Large Barrier to Entry}: open-source software simulators can run on machines researchers already possess,
whereas FPGA-based simulators requires expensive hardware. For instance, the early RAMP prototypes all used
the BEE2 board. While using smaller, more inexpensive boards would reduce initial capital costs, it does so at the expense of reduced simulation capacity.
Simulators using smaller hosts, like RAMPGold, required extensive resource-optimizations to support simulating research-worthy target designs.

\item \emph{Reproducibility of Results}: since RAMP simulators were tied to particular FPGA host
platforms, other researchers would need to purchase the same host to reproduce
publish results. Even if researchers had access to their own FPGAs, they
generally could not run a simulator designed for a different host on their own
FPGAs without modification to simulator.

\item \emph{Modeling Complexity}: designers of RAMP simulators have said that
designing models was more difficult than writing RTL for the matching implementation.
Consider a processor pipeline: to model detailed "cycle-accurate" behavior of that pipeline
the model designer must write RTL that contains much of the same complexity inherent to the actual pipeline's design in order
to properly capture the all hazards that may affect the processors performance.
In order to support modeling a space of different processor designs, either at compile time by generating different model RTL, or at runtime
, by adding logic to permit reconfiguring the simulator, they must add still more complexity.
In order to provide reasonable simulation capacity, they must also apply multi-cycle resource optimizations to the model.
Taken together, this complexity makes the model more difficult to debug than an already-difficult-to-debug FPGA prototype,
and to add insult to injury, these models, like their software counterparts, must still be validated against a real implementation.
\end{itemize}

%
%\section{FAME-1 Target Dataflow Abstractions}\label{sim:fame1-abstractions}
%
%* Another product of the RAMP project and follow on works, were FPGA-friendly abstractions 
%for describing target designs so that they could be easily host-decoupled.
%
%* At a high level, these abstractions represent the target as a dataflow graph of actors. The execution
%of this graph defines a deterministic simulation of the target it represents. These actors can be hosted
%
%* In essence this represents a simplification of conservative Parallel Discrete Event
%Simulation, we will cover this in a later chapter.
%
%in software or on the FPGA -- he
%
%* RAMP model
%  * Used in RAMPGold
%
%* APort Networks
%  * Used in HASim
%
%* Deadlock Avoidance
%
%* LI-BDN


\section{Recent Work at Berkeley}

At the end of the RAMP project it was clear that writing performance models for
FPGAs was not going to be tractable without major breakthroughs in improving
FPGA usability. Since evaluating an RTL implementation with an EDA flow is still
required to meaningfully assess a design's quality-of-result~(QOR), notably it's critical path
delay, effort spent writing FPGA performance models would be better spent
writing the implementation. If the RTL design process could be made more productive,
computer architecture researchers may be more willing to conduct
microarchitecture studies using realizable RTL designs instead of software
simulators. In this model, once an RTL implementation is ready it could be
transformed into a RAMP-like simulator using a \emph{FAME Compiler}. This circumvents the aforementioned model complexity challenge,
as these simulators would essentially be area-optimized hardware emulators of the design:
they would exactly represent the input design's behavior and so would not need
to be validated against silicon.

This was one vision that drove research done by UCB-BAR over the past decade.
To address the RTL design productivity challenge, we developed
Chisel~\cite{Chisel}. Chisel, an RTL design language hosted in Scala, allows
designers to specify rich hardware generators by leaning on the host language to provide
powerful metaprogramming features not available in SystemVerilog or VHDL.
Instead of performing parameter sweeps in a software model, with a generator
the user explores the design space by elaborating and evaluating different
\emph{instances} of the design. The Rocket Chip SoC generator\cite{RocketChip} is one example of this: it generates complete RISC-V SoCs including
core pipelines, private caches, uncores, and common periphery devices. Rocket
Chip lets the user stitch together near-arbitrary networks of disparate devices
using it's diplomacy~\cite{Diplomacy} library, developed by SiFive, which leverages Scala's type
system to provide intelligent, area-optimized system integration. The
Berkeley Out-of-Order Machine (BOOM)~cite{BOOM} and Hwacha vector-fetch
processor~\cite{Hwacha} Chisel generators provide additional core IP that can be
integrated into a Rocket Chip SoC.

To support writing reusable compiler transformations on generated instances, in
Chisel3, Chisel2's internal representation~(IR) of an RTL circuit and its
lowering transforms were replaced with FIRRTL~(Flexible Intermediate
Representation For RTL)\cite{FIRRTL} and its Scala-based compiler. During
elaboration, Chisel3 emits a high-level form of FIRRTL. The FIRRTL compiler
then lowers the instance to Verilog while scheduling user-provided passes. These passes can be used to
a tailor an instance to a particular backend: for instance, when preparing a
design for an ASIC implementation, FIRRTL memories can be replaced
black boxes corresponding to technology-specific SRAMs. Alternatively, if the same
design is destined to become an FPGA prototype, these memories could be double-pumped
to save FPGA resources.

\subsection{Strober and MIDAS}

The first FAME-compiler-like work done at UCB-BAR can be found in the
Strober~\cite{Strober} energy modeling project. While
gate-level-simulation-driven power estimation using commercial tools like
Synopsys PrimeTime PX provides accurate pre-silicon results, gate-level
simulation runs much too slowly to conduct system-level microarchitectural studies using
complete workloads. Instead, Strober generates an accurate average power
estimate by sampling the execution of the workload running on a fast,
FPGA-based simulator and then replaying those samples for short durations in
gate-level simulation. By selecting an appropriate number of samples, average
power dissipation for the SoC can be estimated within a desired error bound.

To realize this, Strober needed to automatically generate an FPGA-based
simulator from ASIC RTL with ability to capture complete RTL state snapshots
and IO traces~\footnote{While these features are available in commercial
emulation tools but no reusable open-source flow existed at the time.}.
Since Strober predated FIRRTL, it modified the Chisel2 backend to:
\begin{enumerate}
    \item Host-decouple input RTL to support halting the simulator to capture
        state snapshots. This is called a \emph{FAME-1 transformation}.
    \item Inject a shadow scan-chain to read out target state.
    \item Elaborate a simulation wrapper around the transformed target. This
        wrapper included IO channels that can buffer a limited IO trace, a
        latency-pipe timing model for the target's DRAM memory system, as well
        as a simulation control bus.
\end{enumerate}

Strober simulators are co-hosted over an FPGA and an CPU -- a \emph{driver}
process, writes to memory mapped registers accessible on the simulation control
bus to advance the simulator and to initiate state snapshot capture.

Initially independent of Strober, MIDAS began as class project to build an
FPGA-based performance simulation framework. It had it's own FAME-1 transform,
and a nascent library of more detailed memory system models~(this would later
become FASED~\cite{FASED}).  Seeing an opportunity to reuse work, we merged
these two projects. MIDAS, as presented at CARRV2018~\cite{MIDAS}, retained all
of Strober's features, but used Chisel3 and FIRRTL, vastly improved simulation
performance, made it easier to support co-simulation of other IO models.

MIDAS was the basis for a number of research projects at UCB-BAR. To improve simulator debuggability, DESSERT~\cite{DESSERT},
leveraged MIDAS's state snapshotting features to perform
\emph{ganged-simulation}. Here one instance of a simulator runs ahead of a second lagging instance, to
detect a simulation error. A detectable error could be either fired hardware assertion, which had been synthesized by the source FIRRTL, or a
commit-log mismatch between the target processor and an online golden model.
On detection of an error, the leading instance instructs the lagging instance to capture a
state snapshot of the target before the error would occur.  Using the replay feature, a
full visibility waveform of the failing target design could then be generated. While
commercial emulators provide rich sets of state-snapshotting features --
DESSERT's differs in that uses two simulators, to let the simulation advance at
full-throughput to the point of failure. Simmani~\cite{Simmani} revisited power
estimation, but instead of using complete state snapshots, injected
a statistically selected set of performance counters into design. These counters
could provide a dynamic estimate of power dissipation. Finally,
MIDAS's DRAM memory models were published as FASED~\cite{FASED}, we cover FASED
in greater detail in a later chapter. For more detail on Strober, MIDAS, and
Simmani, we direct the interested reader to Donggyu Kim's
dissertation~\cite{DGKDissertation}.

\subsection{FireSim}

As Strober was being developed, UCB-BAR continued to study new architectures for
warehouse-scale computers, notably as part of the FireBox~\cite{FireBox} project.
FireBox was an early example of a disaggregated datacenter that relied on
high-bandwidth photonic networking made possible with relatively inexpensive
silicon-integrated photonics, a focus of earlier work done by the lab.  To
simulate FireBox, we built FireSim~\cite{FireSim}. Building on DIABLO, FireSim interconnected MIDAS-generated
simulators with a distributed, CPU-hosted network model to build a
cycle-accurate, warehouse-scale computer simulation. To overcome the usability
challenges of DIABLO, FireSim relies extensively on automation.  Instead of
using a custom host platform, FireSim uses the public cloud, specifically
Amazon Web Service's Elastic Computer Cluster~(EC2).  When the user wishes to run a simulation, FireSim's \emph{manager} program
requests as many FPGAs nodes (to host
Rocket Chip-derived simulators) and compute nodes (to host switch models) as
required. The manager's ability to dynamically spin up and tear down simulators
deployed to EC2 to make it easy to coordinate simulations of arbitrary size. Armed with this flexibility, in our
ISCA2018 publication we were able to reproduce behaviors observable in real
datacenters at a variety of different simulation scales.

Since the 2018 ISCA publication, FireSim has evolved to become a
general-purpose hardware emulation environment for Rocket Chip-based SoCs. By
optionally removing the network simulation, the manager can instead batch out
parallel workloads to multiple independent simulator instances, making it
productive for doing performance evaluations of SoC-scale systems.
Presently, In addition to the manager, FireSim provides:
\begin{enumerate}
    \item A FAME compiler to generate an emulator from FIRRTL. Initially, this
        was MIDAS, but as of version 1.6.0 it has been replaced with Golden Gate~\cite{GoldenGate}.
    \item Device libraries for modeling periphery devices commonly integrated into Rocket Chip SoCs. Including a block device, UART,
        and tether model in addition a NIC model used to performed networked simulation.
    \item FireMarshal, a utility to automate building Linux distributions for
        running on RISC-V based platforms.
    \item Many of the most commonly used RTL libraries in the Rocket-Chip ecosystem, to enable the user to build out interesting systems.
\end{enumerate}

FireSim has been used both in academia and industry, primarily for doing
performance evaluations of new microarchitectural features implemented as
extensions to Rocket Chip.  At Berkeley, FireSim was the basis for
FirePerf~\cite{FirePerf}, which introduced improved instruction tracing, and
non-invasive performance counter integration to perform rapid hardware-software co-design of the NIC
and the Linux networking stack. These features have since been integrated into
mainline FireSim. Academically, FireSim has been used for microarchitectural performance evaluations,
and commercially, it has seen use at Intensivate, Esperanto, and SiFive. FireSim is also being used a RISC-V platform in DARPA's
FETT~\cite{FETT} bug bounty program~\cite{FireSimFETT}.

As a final note, in 2020 we released Chipyard~\cite{Chipyard}, which unifies all of
UCB-BAR's SoC design tools and IP under in a single environment. In Chipyard,
FireSim is a library for doing hardware emulation -- many stand-alone utilities
that were previously hosted in FireSim but aren't strictly related to hardware emulation
support, such as FireMarshal, have been hoisted up into into Chipyard,

\section{Motivations for Golden Gate~(MIDAS II)}

As adoption of FireSim started to increase, it was clear that limitations with
its FAME compiler, MIDAS, prevented it from seeing more widespread use. Firstly, MIDAS's existing FAME-1
transform supported systems with only a single, fixed-frequency clock domain,
and barred use of asynchronous resets. This precludes
simulating most realistic systems and made it challenging to validate FireSim
against existing RISC-V silicon. Secondly, capacity challenges with EC2
FPGAs made it difficult to simulate larger systems, notably the test-chips
UCB-BAR was designing concurrently. Without the ability to do multi-FPGA
partitioning, and without any automated application of RAMP-style
optimizations, this hamstrung FireSim to supporting only relatively small
SoCs. Taken together, these restrictions kneecapped FireSim to being a limited, albeit
open-source, hardware emulation for primitive Rocket Chip-based SoC designs.

To address these two concerns, we set about designing a new FAME compiler
called Golden Gate~\cite{GoldenGate}. Before we expand on the initial
implementation of Golden Gate, in the next chapter we review different target formalisms and
implementation strategies for building FPGA-hosted, discrete-event simulators.

%\section{Segue}
%The ideal full-system simulator -- sufficient for both academic and industrial
%uses -- would be inexpensive, fast and as accurate as desired throughout the
%whole design process; it would always possible to run the software stack on a
%model of the target hardware at speeds fast enough for software development.
%Initially, this simulator could be used for system-level prototyping and design
%space exploration, but as desired, more detailed models or RTL implementations
%of target components would be integrated. Ultimately, once all of the target
%RTL has been integrated, the simulator subsume the role of an FPGA prototype.
%Academics need not completely implement an SoC: they may simply stop adding
%fidelity to the simulation once they are content with the quality of their
%results.
%
%Presently, FPGAs are the only commercial off-the-shelf (COTS) technology
%capable of supporting fast, scalable, cycle-accurate simulation. Thus, we
%believe any attempt to build this ideal simulator must necessarily use FPGAs.
%However, this requires a more flexible perspective of how FPGAs can be deployed as
%simulation \emph{accelerators} and not merely RTL emulation devices as they are
%used in FPGA prototypes.

%\section{Defining FPGA-Accelerated Simulation}
%
%The distinction between FPGA-accelerated simulation and FPGA prototyping can be
%nebulous: we argue that FPGA prototyping represents a narrow subset of the larger
%space of FPGA-accelerated simulation. Key to understanding the distinction is
%first, think of simulation as any other application executing on a host, and
%second, forget that a host may be or include one or more FPGAs.
%
%A simulation is an application that takes an input and produces an output.  The
%output may be the console or file I/O of the target.  Alternatively, it may be
%a dump of the microarchitectural state of the target as it changes over the
%lifetime of the simulation.  As far as the user is concerned, the simulation
%may be optimized in any way whatsoever so long as this output is the same.
%
%Like any other application, simulations have hot spots that account for the bulk
%of their runtime. To improve runtime, the simulation may be parallelized over
%the host, either over multiple homogeneous resources, or by offloading specific
%kernels to accelerators, which may execute concurrently with the rest
%of the application.
%
%%In simulations that account for time at the cycle-level, much of this runtime
%%is dedicated to modeling the cycle-by-cycle interactions of parts of the
%%system. These models may be written in C++ or SystemC, or implemented in an HDL
%%like verilog or VHDL. Parts of the simulation may be divided into functional
%%models, like an architecture simulator, and a timing-model that does some
%%accounting of time based on a model of the microarchitecture. What's important
%%to note, is that any cycle-level model of hardware attempts to capture the
%%behavior of a fine-grained highly concurrent digital-circuit. Taken to the
%%limit, a cycle-level model is an RTL model.
%
%In FPGA-accelerated simulation, in general, we attempt to offload parts of the
%simulation that do cycle-level or cycle-accurate modeling of some part of the
%target. One way to achieve this is to dissolve the simulation spatially
%(perhaps along module boundaries): parts of the target, like an NoC or
%processor pipeline, could be offloaded to an FPGA, while models for I/O may be
%hosted on a CPU.  Alternatively, one could host a functional model of a module
%on the CPU and accelerate a detailed cycle-accurate timing-model on the FPGA
%(or vice versa).  Again, the only constraint on the implementation of the
%simulation, and thus, the implementation of FPGA-accelerated components of the
%simulation, is that its output remains the same. All other things equal, a
%faster implementation is always better.
%
%\section{Compelling Ideas in FPGA-accelerated Simulation}
%
%Researchers have devised many clever techniques for using FPGAs as simulation
%accelerators. To highlight some of these techniques, we summarize the FAME
%taxonomy~\cite{fame}, which outlines three dimensions along which
%FPGA-accelerated simulators may be categorized.
%
%\subsection{FAME-1 (XX1): Host-Target Decoupling}\label{sec:fame1}
%
%In host-decoupled FPGA-accelerated simulators, a target-cycle of simulation
%executes over a variable number of FPGA-host cycles. In contrast, an
%FPGA prototype executes a single target-cycle on every FPGA-host cycle. With
%this technique, ASIC structures that map inefficiently to FPGA fabric may be replaced
%with optimized-for-FPGA structures that take more host cycles to execute, but save
%FPGA resources and improve host-cycle time.  One classic optimization replaces
%multi-ported register files and CAMs with a dual-ported BRAMs pumped over
%multiple cycles.  Additionally, host-decoupling permits the simulator to
%tolerate variable latencies in the host-platform without sacrificing simulator
%performance or changing the target-time behavior of the simulation. Nearly all
%academic FPGA-accelerated simulators are FAME-xx1 simulators, though
%ProtoFlex~\cite{protoflex} is an early example.
%
%\subsection{FAME-2 (X1X): Abstract RTL}
%
%In an abstract-RTL FPGA-accelerated simulation, components of the simulator do
%not model the implementation RTL exactly. Abstraction permits simplifying
%components of the simulator, trading simulation fidelity for FPGA-resource
%savings. Additionally, abstract models can be made reconfigurable in ways the
%implementation RTL cannot. Most academic FPGA-simulators are to some degree
%abstract-RTL simulators.  For example, Strober~\cite{strober} used a
%fixed-latency pipe to model the target's DRAM subsystem~(a FAME-011 model),
%but used FAME-001 models derived directly from source RTL everywhere else.
%Conversely, RAMPGold~\cite{rampgold} was entirely abstract.
%
%\subsection{FAME-4 (1XX): Multithreading}
%
%In a multithreaded FPGA-accelerated simulation, multiple virtual instances of a
%block or module within the target are simulated using a single physical
%datapath on the FPGA. The target state is duplicated according to the number of
%virtual instances, and a scheduler selects which virtual instance should be
%simulated in a given host-cycle. ASIC logic tends to be expensive when mapped
%to FPGA fabric; in FPGA prototypes, designs tend to be logic~(LUT) constrained,
%which leaves much of the FPGA's embedded BRAM left unused.  Multithreading
%improves the mapping efficiency of the target, by reusing the expensive logic
%over multiple copies of target state which can be mapped into abundant FPGA BRAMs and registers.
%HASim~\cite{hasim} and RAMPGold~\cite{rampgold} are examples of simulators that
%employ FAME-x1x multithreading.
%
%
%\subsection{Split Timing-Functional Models}
%
%Splitting timing and functional models is ubiquitous both in software and
%FPGA-accelerated simulators, as it permits amortizing the design effort of the
%functional model over multiple timing models. This is especially crucial in
%FPGA-based simulation, where the functional model can account for the bulk of
%the complexity (e.g., 35K vs. 1K SystemVerilog LoC in RAMPGold~\cite{rampgold}. With
%host-decoupling, it is possible to host the two models on different parts of
%the host-platform: FAST~\cite{fast} hosted its functional model in software and its
%timing model in fabric (ProtoFlex did the opposite~\cite{protoflex}).
%
%For a comprehensive survey of FPGA-accelerated simulation work we direct
%the reader to~\cite{fpgasimbook}.
%
%\section{Adoption Challenges}
%
%Despite their promise, FPGA-accelerated simulators have only been employed by
%the researchers that developed them. The failure to adopt FPGA-accelerated
%simulation methodologies more widely comes as a result of several key factors:
%
%\begin{enumerate}
%
%    \item \textbf{Availability.} Much of the early FPGA-accelerated simulation research
%        relied on boutique FPGA-host platforms like the BEE~\cite{bee2}, or
%        used custom board designs. The cost of these platforms disincentivizes
%        their adoption by researchers who already have the means to run
%        software simulations at low cost.
%
%    \item \textbf{FPGA Capacity.} Common ASIC structures, such as CAMs,
%        multi-ported RAMs, and wide multiplexors are known to map poorly to
%        FPGA fabrics~\cite{fpgagap, fpgagap2}, making it difficult to host
%        large target designs on an FPGA.
%
%    \item \textbf{Configurability \& Extensibility.} Extending FPGA-accelerated
%        simulations requires writing RTL. RTL models are less configurable and
%        harder to extend than software models. Finally, RTL models still need
%        to be validated, further exacerbating the challenge of building them.
%
%    \item \textbf{FPGA compile time.} Compiling an FPGA simulator considerably
%        longer than compiling a software simulator (ones of hours).
%        %To some extent this is inescapable. However, where abstract models are
%        %employed, they can be made run-time configurable, with programmable
%        %registers sitting on a simulation memory map. Where models are
%        %generated from RTL, they can can be incrementally recompiled, or perhaps
%        %partially reconfigured.
%
%    \item \textbf{Debuggability.} Debugging a broken FPGA-accelerated
%        simulation is difficult due to the limited visibility the designer has
%        over the state of the simulation. This is often more challenging than
%        debugging an FPGA prototype of the target, as FPGA-specific
%        optimizations make it more difficult to reason about the state of the
%        target.
%
%\end{enumerate}
%
%\section{Why Continue With FPGA-Accelerated Simulation?}
%
%Even as Moore's law wanes, FPGA capacity continues to scale. The largest FPGAs
%have over 50 MB of BRAM and millions of logic cells\footnote{Comically, scaling
%RAMPGold~\cite{rampgold}, to use the largest Xilinx UltraScale
%FPGA~\cite{ultrascale} by BRAM capacity would permit modeling in excess of 5000
%cores.}. As they have scaled, FPGAs have continued to become more
%heterogeneous, adding features that make them more amenable to hosting
%full-system simulations.  Both Intel and Xilinx now sell FPGAs with embedded ARM
%cores, making it easier to co-simulate tightly coupled hardware and software
%models of a system. Modern FPGAs include hardened DRAM controllers that are
%comparable to those of ASICs. This trend towards greater integration looks to
%continue. For example, upcoming Intel Stratix 10 MX parts include in-package DRAM (HBM2)
%that can support up to 1  of aggregate memory bandwidth~\cite{stratix10mx}.
%Both DRAM capacity and bandwidth are crucial for simulating components
%of the target that may not fit in BRAM.
%
%Lower cost and increased on-chip integration have also made FPGAs more
%accessible. Not only are COTS development boards cheaper and more featureful,
%FPGAs are now available in academic clusters, like TACC's Microsoft
%Catapult~\cite{catapultannounce} deployment, and in datacenters, as a service
%like Amazon Web Services' EC2 F1 instances~\cite{amazonf1}. Where in the past
%academics would have to purchase their own FPGAs -- even to reproduce published
%experiments -- it may soon be possible for them to instead spin up an identical
%simulation on a shared computing resource. Companies may no longer have to
%maintain their own FPGA prototyping clusters; they could instead batch out
%simulations to the cloud.
%
%\section{Improving Usability Through Automation}
%
%While the trends described in the previous section solve the
%\emph{availability} and \emph{FPGA-capacity} challenges, the remaining three
%usability challenges persist. Previous work~\cite{fabscalarfpga, strober} has
%shown that much of an FPGA-accelerated simulator can be automatically generated
%from source RTL. This RTL can be written in an HDL like Verilog or emitted by
%high-level synthesis tools or generators written in languages like
%Chisel~\cite{chisel}. While this still requires an RTL implementation, the same
%RTL can be passed to an EDA flow to measure physical design metrics, and,
%best of all, no validation of the generated model is required.
%
%This is, unfortunately, not a panacea: perhaps the RTL is not yet
%available, or a more abstract, reconfigurable model is desired. In this report
%we consider off-chip DRAM memory systems as a motivating example: they have too
%much state to be hosted in-fabric and yet they must must be tightly coupled to
%the processor model. This makes it difficult to co-simulate DRAM in
%software.\footnote{High-latency peripherals, like disks, can often be modeled
%in software without any performance cost~\cite{disksim}.} These components
%typically require an abstract model that virtualizes the target-memory system
%over FPGA-host-memory system.
%
%This reintroduces the aforementioned problem that anything but a simplistic RTL
%model is difficult to design, modify and reuse. We propose to address this
%through \emph{generators} that synthesize abstract memory system models that
%can be easily modified and used across a wide range of targets.  This generator
%must provide a variety of timing models so as to enable the designer to trade
%fidelity for FPGA-resource savings when needed. Generated models must be reconfigurable, to
%permit sweeping memory system parameters without needing to recompile the FPGA
%bitstream. These models must provide useful instrumentation to both aid in
%debugging and to provide insight about memory system behavior without
%perturbing execution. Finally, when the generator does not provide an
%appropriate timing model, the generator should be easy to extend.
