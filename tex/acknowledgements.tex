In normal times, a Ph.D. is long and often solitary journey, one made possible
by many people not represented in the author strings of talks or publications.
COVID-19 has put this fact on full display.
Suffice it to say, the past year has given me time in abundance to reflect on
those who have made my Ph.D. possible, the difficult times
endurable, and the enjoyable times wonderful.

I must start by thanking the first of my two advisors, Jonathan Bachrach, whose
dynamism had a huge role to play in attracting me to Berkeley in 2014. It's
been a pleasure to work with him on a variety of projects over the years,
including JITPCB. Jonathan's research vision and excitement to realize it is truly
unmatched. Over the years I've tried to import at least some of that into my
own research.  In my final year of graduate school, Jonathan was instrumental
in seeing my through to the end.

During visit days, I remember telling one U of T faculty member I was
interested in working with my future second advisor, Krste Asanovi\'c: he said,
"He likes to build stuff." This remains a fair characterization---Krste and his
students did roll their own ISA after all. It was Krste's, and by osmosis his
group's, desire to build and study real systems that made it so exciting to be
at Berkeley the past seven years. His technical advice has always been
invaluable and, despite being ludicrously oversubscribed in recent years, has
been there when I needed him.

I must thank Sanjit Seshia, whose course on formal verification~(EE219C) was the
most impactful class I took graduate school. It changed how I thought about
verification, and made me appreciate how formal tools could be wielded as a practical
tool for finding bugs. It also gave me the opportunity to convince my friend and
considerably more talented colleague, Albert Magyar, to work on FireSim stuff
with me. Indeed, EE219C's class project was the genesis of Golden Gate, and
Sanjit's and Pramod Subramanyan's guidance in those early days were crucial.

FireSim is a large project made possible by contributions from many people,
including Abe Gonzalez, Albert Ou, Sarah Zhou, Jerry Zhao, Nathan Pemberton,
Dayeol Lee, Emmanuel Amaro, Colin Schmidt, Aditya Chopra,
Qijing Huang, Kyle Kovacs, Tim Snyder, and others still. I'd like to
particularly acknowledge the roles of Howard Mao, whose ability to jump into a gnarly codebase and
get things working has always been source of inspiration for me; Donggyu Kim,
with whom I initially collaborated on MIDAS; Alon Amid, for his tenacity in
attempting to make our tools usable outside of Berkeley and for championing the
Chipyard project; and Sagar Karandikar, who carried FireSim on
his back in the early years, and whose engineering and leadership
sensibilities have strongly informed my own. It's been an enormous pleasure to build out FireSim with you all.

FireSim is one organism in a larger ecosystem of research underway in the UC Berkeley
Architecture Research Group~(UCB-BAR). I'm deeply indebted to the students
who've passed through the group during my time at Cal. Many of the students
that preceded me, including Adam Izraelevitz, Ben Keller, Scott Beamer, Henry
Cook, Yunsup Lee, Chris Celio, Palmer Dabbelt, Brian Zimmer, Angie Wang, and Paul
Rigge, have been mentored me either explicitly or unwittingly, and for that I
am grateful. For many years, Eric Love in particular went above the call of
duty, and took on the role of organizer-in-chief of Krste's group, for this he
deserves a ton of credit. Lisa Wu, one of the rare post-docs in the group, was
always willing to mentor students. I'm thankful for the time she took to coach
me through my first paper presentation. David Bruns-Smith, a fellow climber,
skier, MTGer, and a good friend, was reliably down to walk to get better
coffee---a prerequisite for doing good research. In my latter years, it's been
truly enjoyable to work with younger set of ascendant PhD students, in UCB-BAR
and adjacent groups, including Kevin Laeufer, Vighnesh Iyer, Hasan Genc, Ben
Korpan, and especially Jerry Zhao and Abe Gonzalez, whom I've previously
mentioned for their contributions to FireSim. I leave with a strong sense the
group is in good hands.

Faculty at UC Berkeley have played an important role in getting me to
the finish line. I'd like to thank John Wawrzynek for being my first year
advisor and for welcoming me warmly to Berkeley; Robert Leachman, who
graciously served on my qualification and dissertation committees; William
Kahan and Jim Demmel, for their mentorship on the EDP project; and Bora Nikoli\'c, for being a
pillar of the ASPIRE and ADEPT labs, and for having a better understanding of the
problems FireSim should be solving long before I did.

Additionally I'd like to thank the staff of the ASPIRE and ADEPT Lab, and the
EECS department at large. This includes the current and former administrative
staff of these labs, notably Tami Chouteau, Ria Briggs and previously Roxana
Infante, who have been consistently amazing; Kostadin Ilov, our ever-bouyant
tech admin, who was always eager to tweak our FPGA cluster at my behest; and
Shirley Salanio, who helped me through some of the more challenging times at
Cal, and whose hard work made it possible for me to take Japanese as my outside
minor. It remains one of the great mysteries of the EECS department how Shirley
manages to find the time to be so supportive to so many graduate students.

My journey to graduate school at Cal was born out of relationships in
undergrad. It begins with Rajiv Chopra, who really got me excited about
becoming an academic in the first place. In my fourth year, I had the privledge
to work with Jonathan Rose and his then PhD student, Alex Rodionov. Alex took me
under his wing and laid the groundwork for how I think about designing digital
systems. Jonathan gave me some of the best early advice about navigating
graduate school; I wish I listened to him sooner and more carefully. I also need to thank
many of the engineers I worked with at Altera (now part of Intel) before coming to Cal, notably
Ivan So, Simon So, Byron Sinclair, Elias Ferzli, Ahmed Kammoona, Val
Manohararajah, and most of all, the late Ivan Blunno, to whom I've dedicated
this dissertation. Ivan encouraged me to go to graduate school even though I would've
loved to hack on FPGA CAD tools under his supervision.

Circling back, there are three friends from UCB-BAR I must thank particularly:
\begin{itemize}
\item Jack Koenig. It was a pleasure to be frequent collaborators on some
great~(FASED, Golden Gate) and some not-so-great projects~(EAR, EDP). But hey,
4288 bits to compute a double-precision dot product is a small price to pay to
attend ARITH in London~(thanks Krste). Jack's drive to build awesome-and-usable
tools is infectious, and something I tried to carry into the FireSim project.

\item Andrew Waterman. While his
technical reputation precedes him~(see RISC-V, acknowledgments of other
dissertations), Andrew was my graduate school champion: he believed in me even when I did not.
I can confidently say I would have left in my
third year if it were not for him. I miss our sojourns to Ramen Shop,
our conversations over Japanese-brewed, industrial lagers and the hours
``lost" to Halo thereafter.

\item Albert ``The Man" Magyar. I've had the pleasure of working with Albert at
various points of graduate school including on EAR~(the ``gratuitous
amounts of cache" project), but our close collaboration on Golden Gate
defined the latter half of my graduate school experience. Golden Gate
was built on spontaneous conversations in the Cosmic Cube and in afternoon
voyages to Yali's cafe. It's truly unfortunate that COVID-19 robbed us of these interactions in our
final year of graduate school, and my research suffered for it. If I'm
being honest, I mostly missed his wit.
\end{itemize}

While at times my research was nearly all-consuming, friendships outside of my
research circles provided some critical balance in my life. I have many fond
memories from early graduate school with friends made during visit days, namely
George and my first roommate, Alyssa. Rock climbing kept me sane over the past
seven years with Berkeley Ironworks serving as a vital escape. Shout out to
my partners over the years: Pascal, Nick, Josh, and David. I must thank Simon,
Claire, and particularly, Michael for putting together some wonderful
backpacking trips. I'd also like to thank my latest roommate, Gabriel, for
welcoming me into his life, getting me excited about baking, and for showing me
the true versatility of .

When times were rough at Berkeley, I often looked to friends from Toronto
including Matt, Denise, Judith, Connor, Mohamed, Marissa, Laurence, Cindy, and
others. I'd like to call out a few in particular: Kevin, his time spent in SF
was a wonderful period of graduate school for me, I miss having the excuse to
visit the city; Zimu Zhu, for always checking in with me despite my online
reticence, it was hard living on a separate coast from him; Richard, for
providing some of my fondest memories of graduate school forged in trips to San
Deigo and Scotland; and finally, Kelvin: my stalwart high school friend. I
struggle to articulate how grateful I am he came to Cal for his Ph.D.

I conclude by acknowledging the two constants in my life. First, my family,
namely Mum, Dad, and my sisters, Emma and Laura---it's been a long seven years
and they've been behind me the whole way. And second, Fiona, who's reliably put
up with my sh*t over seven years of a long distance relationship, and, through
some mysterious alchemy, can always coax a smile out of me. Next time I hope we
can agree to get admitted to the same grad school.
