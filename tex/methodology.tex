In section~\ref{sec:framework}, we described how MIDAS takes a target and
maps it to a host. Here, we describe the microarchitecture of the target
machines we simulate, give an overview of the SPEC2017 Integer benchmarks that
we run in our evaluation, and explain how we instrument our target designs.

\subsection{Target Designs}\label{sec:target-parameters} Our target designs 
are derived from the Rocket Chip generator~\cite{rocketchip}, which contains
Rocket, a single-issue in-order scalar core implementing the RISC-V
ISA~\cite{Waterman:EECS-2016-118, Waterman:EECS-2016-161}~(RV64IMAFDC). Rocket Chip
has been taped out over a dozen times for both research and commercial
purposes. In our
experiments, we use the default
configuration of Rocket Chip, which includes
a \wunits{16}{KiB} L1 I\$, a blocking \wunits{16}{KiB} L1 D\$, and 32-entry
fully-associative L1 I and D TLBs. We only change the default configuration to
increase the number of performance counters and deepen the L2 TLB to 1024 entries.

At the system level, our targets consist of single or quad-core instantiations
of Rocket Chip (labeled SC or QC), composed with either a latency-bandwidth
pipe~(labeled LBP), or a quadruple rank DDR3-2133~(14-14-14) FCFS or FR-FCFS
DRAM models over a 64-bit AXI4 bus. In all targets, the simulated system has
16 GiB of DRAM capacity. Additionally, two targets include a 4 MiB
LLC model~(labeled LLC). In the target's periphery, we have a UART and block
device that interact with simulation models co-hosted in software.

\begin{table*}
\centering
    \begin{tabular}{|c|c|c|c|c|c|c|c|c|c|c|c|c|}
	\hline
        \textbf{Resource} & \multicolumn{4}{|c|}{Logic LUTs} &
        \multicolumn{4}{|c|}{Registers} &
        \multicolumn{4}{|c|}{36K BRAMs} \\
	\hline
	\hline
        \textbf{Target Design} &
        \multicolumn{2}{|c|}{Simulator} &
        \multicolumn{2}{|c|}{Memory Model} &
        \multicolumn{2}{|c|}{Simulator} &
        \multicolumn{2}{|c|}{Memory Model} &
        \multicolumn{2}{|c|}{Simulator} &
        \multicolumn{2}{|c|}{Memory Model} \\
	\hline
    SC-FCFS & 48664 & 5.3 \% & 4680 & 0.5 \% & 24608 & 1.3 \% & 3071 & 0.2 \% & 30 & 1.8 \% & 4 & 0.2 \% \\
SC-FRCFS & 50638 & 5.5 \% & 6095 & 0.7 \% & 25019 & 1.4 \% & 3565 & 0.2 \% & 30 & 1.8 \% & 4 & 0.2 \% \\
SC-LLC-FCFS & 49969 & 5.5 \% & 6158 & 0.7 \% & 24989 & 1.4 \% & 3536 & 0.2 \% & 70 & 4.2 \% & 44 & 2.6 \% \\
QC-FRCFS & 128714 & 14.1 \% & 6014 & 0.7 \% & 62480 & 3.4 \% & 3565 & 0.2 \% & 100 & 6.0 \% & 4 & 0.2 \% \\
QC-LLC-FRFCFS & 130329 & 14.3 \% & 7747 & 0.8 \% & 62870 & 3.4 \% & 3957 & 0.2 \% & 140 & 8.3 \% & 44 & 2.6 \% \\

	\hline
	\end{tabular}

    \caption{XCVU9P resource utilization for a space of different targets.
    Percentages indicate the share of total FPGA resources consumed by that
    design partition. Simulator totals are inclusive of the memory model.}
\label{tbl:utilization}
\vspace{-0.1in}
\end{table*}

We report the utilization of several host-mapped targets in
Table~\ref{tbl:utilization}.  We separate the utilization contributions into ``Simulator''~(the
component of the design generated by MIDAS, including the \PNAME instance),
and Memory Model~(only the \PNAME instance). Not shown is the contribution of the
AWS shell, which consumes a constant 14.5\%, 10\%, and 16.7\%  of the XCVU9P's
Logic LUT, Register, and 36K BRAM resources respectively.

\subsection{System Software}

All benchmarks were run on Linux kernel version 4.15.0-rc6. We built base Linux
distributions with Buildroot and BusyBox. As part of our FPGA batch-job submission
scripts, these base images are modified to add the desired workload and to run
the workload immediately after Linux boot by altering the \texttt{init}
script. During simulation, target processes pipe standard out to the target
filesystem. We retrieve these files by remounting the filesystem on the
host-CPU after the simulation has completed.

\subsection{Instrumentation}
To measure core-side performance counters, we run a target program that, on a
one-target-second timer interrupt~(1 billion cycles), reads a core's
performance counters and dumps them to the target filesystem. We pin an
instance of this program to each core. To measure DRAM-side statistics,
we pause the simulator every one-billion target cycles and read out the
memory-mapped instrumentation registers. Unlike using a target program to
obtain these values, this approach does not alter the target's behavior.

\subsection{An Overview SPEC2017int On Rocket}
In our experiments, we run SPEC2017 intrate and intspeed suites with reference
inputs, cross-compiled for RISC-V systems with the \texttt{-O2} flag.
Intspeed benchmarks require as much as \wunits{16}{GB} of memory, while intrate
benchmarks require \wunits{2}{GB} per copy.  In Table~\ref{tbl:spec-intspeed}
and Table~\ref{tbl:spec-intrate}, we give each suites' dynamic instruction
count and L1 MPKIs when running on the Rocket configuration described in
Section~\ref{sec:target-parameters}

\begin{table}[t]
\centering
    \begin{tabular}{|c|c|c|c|}
    \hline
        \textbf{Benchmarks} & \textbf{Instructions~(T)} & \textbf{D\$ MPKI} & \textbf{I\$ MPKI} \\
    \hline
        600.perlbench & 2.98 & 9.0 & 10.0 \\
        602.gcc & 2.43 & 36.6 & 9.7 \\
        605.mcf & 1.60 & 97.9 & 0.1 \\
        620.omnetpp & 1.11 & 56.9 & 9.3 \\
        623.xalancbmk & 1.21 & 62.9 & 7.9 \\
        625.x264 & 4.55 & 3.0 & 2.9 \\
        631.deepsjeng & 2.51 & 8.7 & 15.4 \\
        641.leela & 2.59 & 5.8 & 1.5 \\
        648.exchange2 & 3.24 & 0.0 & 0.1 \\
        657.xz & 9.41 & 19.8 & 0.2 \\
    \hline
    \end{tabular}
    \caption{Dynamic instruction counts and MPKI on Rocket running SPEC2017 intspeed benchmark~(single threaded).}
    \label{tbl:spec-intspeed}
\vspace{-0.1in}
\end{table}

\begin{table}[t]
\centering
    \begin{tabular}{|c|c|c|c|}
    \hline
        \textbf{Benchmarks} & \textbf{Instructions~(T)} & \textbf{D\$ MPKI} & \textbf{I\$ MPKI} \\
    \hline
        500.perlbench & 2.99 & 8.9 & 10.1 \\
        502.gcc & 1.35 & 29.5 & 11.1 \\
        505.mcf & 0.91 & 80.9 & 0.1 \\
        520.omnetpp & 1.11 & 56.6 & 10.4 \\
        523.xalancbmk & 1.21 & 62.9 & 7.6 \\
        525.x264 & 4.55 & 3.0 & 3.0 \\
        531.deepsjeng & 2.14 & 8.2 & 15.3 \\
        541.leela & 2.59 & 5.8 & 1.5 \\
        548.exchange2 & 3.24 & 0.0 & 0.1 \\
        557.xz & 2.25 & 15.7 & 0.1 \\
    \hline
    \end{tabular}
    \caption{Dynamic instruction counts and MPKI on running the SPEC2017 intrate benchmark~(single copy).}
    \label{tbl:spec-intrate}
\vspace{-0.1in}
\end{table}
