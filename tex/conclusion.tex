Our goal for the FireSim project was to build a radically
inexpensive, yet fast and productive, cycle-accurate full-system simulation
technology. This was in service of reducing key contributor to the NRE of building silicon.  While
FireSim is most similar to existing hardware emulators, our approach is
unique in that is uses a single commercial-off-the-shelf FPGA, and depends on a
completely open-source toolchain. However, it was difficult to make a comparison to existing hardware emulation solutions
in good faith, as early versions of FireSim had two critical limitations: the SoCs it could support 
were small, and could possess only a single-clock domain.

We designed Golden Gate to address these limitations~(Chapter~\ref{sec:golden-gate}). First, Golden Gate uses a LI-BDN
target formalism and RAMP-inspired multi-cycle optimizations to fit larger SoCs
on a single FPGA. Descriptions of these optimizations can be found in Albert
Magyar's dissertation~\cite{MagyarDissertation}. Overcoming the clocking limitations was the primary focus of this dissertation.
To model multiple clock domains, we introduced a new FAME transform that avoids
using dedicated FPGA clocking resources like prior academic work~\cite{DVFSPrototype}, in favor of a
clock-gating scheme that should be portable to many different FPGA
platforms~(Chapter~\ref{sec:static-multiclock}). To model clock switching and
generation circuits, which form the basis of dynamic frequency scaling support
in realistic SoCs, we introduced a timestamped subgraph into the simulator that
implements a conservative PDES~(Chapter~\ref{sec:dynamic-multiclock}).

One take-away from our work in supporting clock switching and generation
structures is that closely coupled circuits, like ICGs and
clock multiplexors, are probably best simulated \emph{in situ}, instead of
being extracted into a decoupled unit. State for these circuits can be left in
the target, and combinational functions they perform on clocks can be hoisted
into the hub unit's second stage. Here they would act directly on future clock
enables to drive additional clock buffers. Clearly, this would be
insufficient for modeling clock generators, like PLLs, that can't be described
as combinational functions on existing clocks. For modeling circuits of this
nature, having an independent TUs seems entirely sensible.  This hybrid
approach would also reduce the number of timestamped inputs on the hub unit,
addressing a potential scalability challenge in systems with many clocks.
Implementing and studying this approach is the logical continuation of the
work described in this dissertation.

While Golden Gate has made inroads in simulating far more realistic SoCs on a
single FPGA, there are still many domains under the FireSim project that
require attention.  First, FireSim should be ported to other FPGAs to verify
our claims about the flexibility of our approaches. Here there are ongoing
efforts both at Berkeley and from the community at large. Perhaps the biggest
frontier for innovation lies in improving FireSim's debuggability. Snapshotting
features, to provide greater visibility over the target, are important feature
in commercial hardware emulators that FireSim currently lacks. MIDAS did have
support for this, but its implementation presupposed a monolithic, single clock
domain hub.  Supporting resource-efficient state capture that co-exists with
resource optimizations is both an important and compelling avenue for future
work.

FireSim's growing user base, both academic and industrial, suggests our vision
for a more cost-effective full-system simulation technology addresses a
material technology gap. We hope that FireSim and the contributions of this
dissertation inspire more academic research in this domain in the future.
