Our goal in building out the FireSim project was to build a radically
inexpensive, yet fast and productive, cycle-accurate full-system simulation
technology to attack a key contributor to the NRE of building silicon.  While
our system is most similar to existing hardware emulators, our approach is
unique in that is uses a single COTS FPGA, and depends on a completely open
source toolchain. However, it was difficult to make this comparison in good
faith, as early versions of FireSim had two critical limitations: were limited
to supporting to small, single-clock domain.

We designed Golden Gate to address these limitations~(Chapter~\ref{sec:golden-gate}). First, Golden Gate uses a LI-BDN
target formalism and RAMP-inspired multi-cycle optimizations to fit larger SoCs
on a single FPGA; descriptions of these optimizations can be found in Albert
Magyar's dissertation. Overcoming the clocking limitations was the primary focus of this dissertation.
To model multiple clock domains, we introduced a new FAME transform that avoid
using dedicated FPGA clock resources like prior academic work, in favor of a
clock-gating scheme that should be portable to many different FPGA
platforms~(Chapter~\ref{sec:static-multiclock}.  To model clock switching an
generation circuits, which form the basis of dynamic frequency scaling support
in realistic SoCs, we introduced a timestamped subgraph into the simulator that
implements a conservative PDES~(Chapter~\ref{sec:dynamic-multiclock}).

One takeaway from our work in supporting clock switching and generation
structures is that closely coupled clock-generating circuits, like ICGs and
clock multiplexors, are probably best simulated \emph{in situ}, instead of
being extracted into a decoupled unit. State for these circuits can be left in
the target, and combinational functions they perform on clocks can be hoisted
into the hub-unit's second stage to act on future clock enables. This would
generate new clock buffers for these derived clocks. This clearly would be
insufficient for modelling clock generators like PLLs that can't be described
as digital functions on existing clocks, like PLLs.  For modelling this class
of circuits, having an independent TU unit seems entirely sensible.  This
hybrid approach would also reduce the number of timestamped inputs on the hub
unit, addressing a potential scalability challenge in systems with many clocks.
Implementing and studying this approach is the logical continutation of the
work described in this dissertation.

While Golden Gate has made inroads in simulating far more realistic SoCs in
FireSim, there are still many domains under the FireSim project that require
attention.  First, FireSim should be ported to other FPGAs to verify our claims
about the flexibility of our approaches. Here there are ongoing efforts, both
at Berkeley and abroad. Perhaps the biggest frontier for innovation lies in
improving FireSim's debuggability. Snapshotting features, to provide greater
visibility over the target, are important tool in commerical hardware
emulators. Earlier verions of MIDAS had support for this, but that
implementation presupposed a monolithic, single-clock domain hub and used a
scan chain that increased FPGA resource utilization.  Supporting
resource-efficient state capture that co-exists with resource optimizations is
both an important and compelling avenue for future work.

FireSim's growing userbase of both industrial and academic users, suggests our
vision for a more cost-effective full-system simulation technology addresses a
material technology gap. We hope that FireSim and the contributions of this
dissertation inspire more academic research in to better open-source hardware
emulation systems in the future.
